\input{preambuloSimple.tex}

%----------------------------------------------------------------------------------------
%	TÍTULO Y DATOS DEL ALUMNO
%----------------------------------------------------------------------------------------

\title{	
	\normalfont \normalsize 
	\textsc{\textbf{Planificación y Gestión de Proyectos Informáticos (2018-2019)} \\ Máster Profesional de Ingeniería Informática \\ Universidad de Granada} \\ [25pt] % Your university, school and/or department name(s)
	\horrule{0.5pt} \\[0.4cm] % Thin top horizontal rule
	\huge Aplicación de Detección Temprana de Fracturas mediante Resonancias \\ % The assignment title
	\horrule{2pt} \\[0.5cm] % Thick bottom horizontal rule
}

\author{Alejandro Campoy Nieves \\ Luis Gallego Quero} % Nombre y apellidos y correo
\date{\normalsize\today} % Incluye la fecha actual
\usepackage{graphicx}
\usepackage{hyperref} % Para añadir los hiperenlaces.


%----------------------------------------------------------------------------------------
% DOCUMENTO
%----------------------------------------------------------------------------------------

\begin{document}
\maketitle % Muestra el Título

\newpage %inserta un salto de página

\tableofcontents % para generar el índice de contenidos

%\listoffigures

%\listoftables	

\newpage		
 
\section{Resumen}

El proyecto en el que vamos a trabajar trata sobre el desarrollo de una aplicación capaz de analizar un gran almacén de imágenes de resonancias magnéticas mediante alguna librería Deep Learning, proporcionando al marco sanitario un reconocimiento de fracturas sin la necesidad de la supervisión de un profesional. Para ello, el diagnóstico se mostrará directamente en la pantalla del sistema sanitario, con tal porcentaje de éxito que la evaluación del médico asignado no será apenas necesaria.  \\

Además, la aplicación supondrá un gran avance en el diagnóstico precoz de fracturas, ya que mediante el entrenamiento continuo del reconocimiento de fracturas y de indicios previos a esta, se podrá detectar con mucha seguridad dichas posibles fracturas. Para ello, una vez obtenido un alto porcentaje de éxito en la resolución de las fracturas, se trabajará con un paso previo, analizando resonancias previas a la fractura en si. 

\section{Lugar de Ejecución}

Inicialmente la propuesta está enfocada al Parque Tecnológico de la Salud (PTS) de Granada. Una vez integrado correctamente con el sistema médico del PTS, se estudiarán y evaluarán las posibilidades de expansión para futuros hospitales que requieran este servicio. Sin duda una meta muy importante es terminar instalando en cualquier centro médico de España.

\section{Objetivos}
\subsection{Generales} 
\begin{itemize}
	\item Mejorar el desempeño y eficiencia de los profesionales.
	\item Ahorro de tiempo en el diagnóstico.
	\item Incrementar la tasa de acierto en el diagnóstico.
	\item Introducir técnicas de Deep Learning en los hospitales.
	\item Mejora de la calidad sanitaria.
\end{itemize}

\subsection{Específicos}
\begin{itemize}
	\item Detección fiable de fracturas.
	\item Detección precoz de fracturas, siendo estas aún pequeñas fisuras o esguinces.
	\item Tasa de fallo inferior al 1\% en los falsos negativos.
	\item Eficiencia en el proceso de análisis, siendo posible realizar diagnósticos en pocos segundos.
\end{itemize}

\section{Antecedentes}

El personal que trabajará en el proyecto es el siguiente:
\begin{itemize}
	\item \textbf{Luis Gallego Quero}: Graduado en Ingeniería Informática. Cuenta con gran experiencia en el desarrollo de aplicaciones especialmente web, siendo esto de gran utilidad a la hora del desarrollo de toda la parte de comunicación del software.
	\item \textbf{Alejandro Campoy Nieves}:  Graduado en Ingeniería Informática. Cuenta con conocimiento en algoritmos de clasificación, estrategias de búsqueda basadas en inteligencia artificial y en ingeniería del conocimiento.
\end{itemize}

Sin duda la unión del desarrollo de software con la Inteligencia Artificial orientada al Deep Learning es esencial para correcto desarrollo del proyecto que abarcamos, culminando sin problemas en una realización satisfactoria. \\

El grupo está formado por estudiantes de la Escuela Técnica Superior de Ingeniería Informática y Telecomunicaciones con estudios de Ingeniero Técnico Informático. Por ello, la única experiencia actual reside en proyectos anteriores realizados que están relacionados con el proyecto en cuestión. Por ejemplo, la implementación de algunas estrategias de inteligencia artificial para solucionar distintos problemas como podría ser la clasificación de ciertos objetos y prácticas relacionadas con la visión por computador. Además, la experiencia en desarrollo de aplicaciones será clave para la integración del modelo desarrollado con la aplicación que usan en el sistema sanitario.

\section{Justificación}

Una vez que hemos dado forma a la idea del proyectos que buscamos construir, hemos completado un estudio de mercado para comprobar la competencia a la que nos enfrentamos. Para ello hemos realizado una búsqueda de trabajos similares o estrechamente relacionados que puedan ser relevantes para este proyecto. Entre los resultados obtenidos, hemos encontrado algunas aplicaciones de interés como un algoritmo de detección de \href{https://blogthinkbig.com/este-algoritmo-diagnostica-neumonia-con-la-precision-de-un-medico}{neumonías} \cite{article:neumonia} o una \href{https://www.face2gene.com/}{aplicación} \cite{misc:face2gene} que realiza evaluaciones genéticas completas y precisas a partir de un reconocimiento facial. \\

También como documentación interesante hay que tener en cuenta la viabilidad del proyecto y obtener el conocimiento sobre el problema que se pretende resolver mediante Deep Learning. Este \href{https://blog.hospitalsanangelinn.mx/resonancia-magnetica-diagnostico}{enlace} \cite{article:resonancia} nos ofrece información de qué cosas se pueden detectar mediante una resonancia magnética teniendo conocimiento suficiente para poder interpretarla. \\

Un ejemplo de fractura por tensión perfectamente identificado sería el que aparece en la \href{https://www.google.es/search?q=resonancia+magnetica+con+fractura&source=lnms&tbm=isch&sa=X&ved=0ahUKEwj_ydaisK7eAhVDgRoKHeGoDsUQ_AUIDigB&biw=1536&bih=754#imgrc=5wcLStUv22LjAM:}{imagen de referencia} \cite{misc:imagen}.

\section{Innovación}

Para innovar nos vamos a basar en una de las nuevas tecnologías emergentes, que ha pasado de ser una moda a una realidad capaz de solucionar problemas reales, siendo esta Deep Learning. Utilizando esta técnica, queremos agilizar los diagnósticos médicos proporcionando directamente un resultado de la resonancia sin necesidad de llevar a cabo un estudio por parte de un profesional. \\

Además, el amplio abanico de posibilidades que abre tras el desarrollo de un sistema así son muy grandes. Basándonos en esta idea, se podría detectar muchas otras causas o enfermedades que puedan ser representadas en imágenes, lo cual consideramos que proporciona un gran avance en innovación.

\section{Actividades a realizar alineadas con los objetivos}

Siendo esta aún una fase inicial del proyecto, hemos pensado en una serie de actividades a realizar con la finalidad de obtener los objetivos previstos:

\begin{itemize}
	\item Estudio del problema. Para obtener conocimiento experto sobre los algoritmos Deep Learning y sobre el reconocimiento fracturas.
	\item Análisis de parámetros e información de entrada a tener en cuenta.
	\item Desarrollo de un algoritmo de aprendizaje Deep Learning. Esto es solamente una opción, ya que es muy interesante utilizar algoritmos ya desarrollados por empresas como Google.
	\item Entrenamiento de la red neuronal con ejemplos de resonancias.
	\item Prueba del modelo entrenado con resonancias exteriores al conjunto de entrenamiento.
\end{itemize}

\section{Cauces de seguimiento}

Inicialmente se realizará una reunión con el cliente en la que trataremos todos los requisitos necesarios para el desarrollo, obteniendo así un análisis completo del sistema. Posteriormente, cuando haya una base construida se podrán realizar reuniones periódicas con el cliente proporcionándole diseños de los nuevos módulos que se vayan a construir a la vez que se evaluará el estado de la aplicación. Entendemos por esto último la muestra de la eficiencia del algoritmo a la hora de reconocer las fracturas, en definitiva, porcentajes de acierto.\\

Estas reuniones serán vitales en el desarrollo ya que la puesta en común del estado del sistema nos proporcionará la oportunidad de encontrar fallos y de darnos cuenta de situaciones en la que se pueden mejorar.

\section{Valor añadido}

Dentro de los avances que supone el proyecto que abarcamos, podemos encontrar una mejora de eficiencia y calidad en el diagnóstico. De esta forma, ganamos en seguridad de los resultados obtenidos al mismo tiempo que mejoramos la rapidez en la que se atiende a pacientes con este tipo de problemas de salud. Finalmente, como valor añadido extra, se podrá diagnosticar fracturas incluso antes de que estas ocurran.

\section{Beneficios y beneficiarios}

En este proyecto distinguimos dos beneficiarios principalmente, con una serie de beneficios distintos aunque similares para cada uno. \\

Por un lado, los pacientes, como ya se ha mencionado anteriormente, consiguen una mejora en la calidad de los diagnósticos que se les realiza, incrementando la tasa de acierto al mismo tiempo que no tienen la necesidad de esperar a ser evaluados por un experto cualificado. \\

Por otro lado, los hospitales que implementaran este sistema, verían una mejora económica debido a la disminución de recursos necesarios y tiempo. De tal forma que tendrían a esos mismos pacientes con una mejor calidad de servicio.


\bibliographystyle{plain}
\bibliography{biblio}


\end{document}       
%---------------------------------------------------
