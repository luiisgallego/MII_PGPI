\input{preambuloSimple.tex}
%	TÍTULO Y DATOS DEL ALUMNO
%----------------------------------------------------------------------------------------

\title{	
	\normalfont \normalsize 
	\textsc{\textbf{Planificación y Gestión de Proyectos Informáticos (2018-2019)} \\ Máster Profesional de Ingeniería Informática \\ Universidad de Granada} \\ [25pt] % Your university, school and/or department name(s)
	\horrule{0.5pt} \\[0.4cm] % Thin top horizontal rule
	\huge Práctica A-I \\ % The assignment title
	\horrule{2pt} \\[0.5cm] % Thick bottom horizontal rule
}

\author{Alejandro Campoy Nieves \\ Luis Gallego Quero} % Nombre y apellidos y correo
\date{\normalsize\today} % Incluye la fecha actual

\usepackage[spanish, es-tabla]{babel}
\usepackage{hyperref} % Para añadir los hiperenlaces.
\hypersetup{
	colorlinks=true,
	linkcolor=blue,
	filecolor=magenta,      
	urlcolor=cyan,
}
\usepackage{graphicx}
\usepackage{amssymb, amsmath, amsbsy}
\usepackage{mathptmx}	
\usepackage{float}
\usepackage{booktabs}					%paquete para realización de tablas profesionales
\usepackage{eurosym}
%\usepackage[table]{xcolor}
%\definecolor{lightgray}{gray}{0.9}
\usepackage{xcolor}
\usepackage{colortbl}


%----------------------------------------------------------------------------------------
% DOCUMENTO
%----------------------------------------------------------------------------------------

\begin{document}
	\maketitle % Muestra el Título
	
	\newpage %inserta un salto de página
	
	\tableofcontents % para generar el índice de contenidos
	
	\listoffigures
	
	\listoftables	
	
	\newpage	
 
\section{Restricciones de negocio}


\begin{table}[H]
	\begin{center}
		\begin{tabular}{|l|l|l|}
			\hline 
			Restricción de negocio & Información & Comentarios \\ 
			\hline \hline
			Presupuesto & 118000 euros & Depende de varias subvenciones \\ \hline
			Plazo máximo proyecto & hasta Junio & \\ \hline
			Cobertura mínima & Necesitamos darle más prioridad  & Gran cantidad de los fondos proviene de éste \\ 
			 & a la cobertura del laboratorio &  \\ \hline
			Marca de los dispositivos de red & Marcas hardware Juniper & Cisco no es válido\\ \hline
			Límite construcción de la red & De 2 meses & Migración de datos prevista \\ \hline
		\end{tabular}
		\caption{Tabla de Restricciones de negocio.}
		\label{tabla:tabla1}
	\end{center}
\end{table}

\section{Restricciones técnicas}

\begin{table}[H]
	\begin{center}
		\begin{tabular}{|l|l|l|}
			\hline 
			Restricción técnica & Información & Comentarios \\ 
			\hline \hline
			Cobertura de la red & Al menos, en las plantas 4, 6 y 7 & Sería conveniente extenderlo al resto del hospital \\ \hline
			Compatibilidad & Se busca que el proyecto sea compatible con lo configurado anteriormente & \\ \hline
			Software libre &  & \\ \hline
			Instalación remota & Ordenadores se instalan cada semana & Los viernes a las 21:00 \\ \hline
			Muestras telepatología & Cada muestra pesa entre 2 y 4 GB & \\ \hline		
		\end{tabular}
		\caption{Restricciones técnicas.}
		\label{tabla:tabla2}
	\end{center}
\end{table}

\section{Características básicas de las aplicaciones de la red}

\begin{table}[H]
	\begin{center}
		\begin{tabular}{|l|l|l|l|l|l|}
			\hline 
			Nombre & Tipo & ¿Nueva? & ¿Es crítica? & Localización & Comentarios  \\ 
			\hline \hline
			WakeOnLan & & & & & \\ \hline
			Rembo & & & & & \\ \hline
			Windows 7 & & & & & \\ \hline
			LibreOffice & & & & &  \\ \hline
			Navegador web & & & & &  \\ \hline
			QuPath & & & & & Herramienta para visualizar y analizar imágenes de muestras digitalizadas en el navegador web y compartirlo con otros especialistas \\ \hline 
		\end{tabular}
		\caption{Características básicas de las aplicaciones de la red.}
		\label{tabla:tabla3}
	\end{center}
\end{table}

\section{Requisitos técnicos de las aplicaciones}

\begin{table}[H]
	\begin{center}
		\begin{tabular}{|l|l|l|l|l|}
			\hline 
			Nombre & MTBF/MTTR & Coste de parada del servicio & Tasa transferencia requerida & Latencia requerida \\ 
			\hline \hline
			& & & & \\ \hline
			& & & & \\ \hline
			& & & & \\ \hline
			& & & &  \\ \hline
		\end{tabular}
		\caption{Requisitos técnicos de las aplicaciones.}
		\label{tabla:tabla4}
	\end{center}
\end{table}


\section{Tabla de caracterización de usuarios}

\begin{table}[H]
	\begin{center}
		\begin{tabular}{|l|l|l|l|}
			\hline 
			Nombre de la comunidad de usuarios & Nombre de miembros & Localización & Aplicaciones usadas \\ 
			\hline \hline
			Recepcionista & & Recepción  & \\ \hline
			Directora & Aurora & Despacho de dirección  & \\ \hline
			 & 1 persona, sin nombre & Servicio de informática & \\ \hline
		    Secretaría & 2 personas, sin nombre  & &  \\ \hline
		    Centro de datos & 1 persona, sin nombre  & Centro de datos &  \\ \hline
		    Séptima planta, sala de espera & Arturo  & Consultas &  \\ \hline
		    Cuarta planta, consultas & 1 persona, sin nombre  & Consultas planta cuarta & Historial clínico  \\ \hline
		    Primera sala de telepatología & 1 persona, sin nombre & Sala 1 telepatología & QuPath
		\end{tabular}
		\caption{Tabla de caracterización de usuarios.}
		\label{tabla:tabla5}<ç
	\end{center}
\end{table}

\section{Tabla de características del tráfico de red generado por las aplicaciones}

\begin{table}[H]
	\begin{center}
		\begin{tabular}{|l|l|l|l|l|l|l|}
			\hline 
			Aplicación & Tipo de flujo de tráfico & Protocolos usados & Comunidades & Almacenes de datos & Ancho de banda requerido & QoS \\ 
			\hline \hline
			& & & & & & \\ \hline
			& & & & & & \\ \hline
			& & & & & & \\ \hline
			& & & & & &  \\ \hline
		\end{tabular}
		\caption{Tabla de características del tráfico de red generado por las aplicaciones.}
		\label{tabla:tabla6}
	\end{center}
\end{table}

%\newpage
%\bibliographystyle{plain}
%\bibliography{biblio}

\end{document}       
%---------------------------------------------------
