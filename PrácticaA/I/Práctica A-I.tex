%%%%%%%%%%%%%%%%%%%%%%%%%%%%%%%%%%%%%%%%%
% Short Sectioned Assignment LaTeX Template Version 1.0 (5/5/12)
% This template has been downloaded from: http://www.LaTeXTemplates.com
% Original author:  Frits Wenneker (http://www.howtotex.com)
% License: CC BY-NC-SA 3.0 (http://creativecommons.org/licenses/by-nc-sa/3.0/)
%%%%%%%%%%%%%%%%%%%%%%%%%%%%%%%%%%%%%%%%%

%----------------------------------------------------------------------------------------
%	PACKAGES AND OTHER DOCUMENT CONFIGURATIONS
%----------------------------------------------------------------------------------------

\documentclass[paper=a4, fontsize=11pt]{scrartcl} % A4 paper and 11pt font size

% ---- Entrada y salida de texto -----

\usepackage[T1]{fontenc} % Use 8-bit encoding that has 256 glyphs
\usepackage[utf8]{inputenc}
%\usepackage{fourier} % Use the Adobe Utopia font for the document - comment this line to return to the LaTeX default

% ---- Idioma --------

\usepackage[spanish, es-tabla]{babel} % Selecciona el español para palabras introducidas automáticamente, p.ej. "septiembre" en la fecha y especifica que se use la palabra Tabla en vez de Cuadro

% ---- Otros paquetes ----

\usepackage{url} % ,href} %para incluir URLs e hipervínculos dentro del texto (aunque hay que instalar href)
\usepackage{amsmath,amsfonts,amsthm} % Math packages
%\usepackage{graphics,graphicx, floatrow} %para incluir imágenes y notas en las imágenes
\usepackage{graphics,graphicx, float} %para incluir imágenes y colocarlas

% Para hacer tablas comlejas
%\usepackage{multirow}
%\usepackage{threeparttable}

%\usepackage{sectsty} % Allows customizing section commands
%\allsectionsfont{\centering \normalfont\scshape} % Make all sections centered, the default font and small caps

\usepackage{fancyhdr} % Custom headers and footers
\pagestyle{fancyplain} % Makes all pages in the document conform to the custom headers and footers
\fancyhead{} % No page header - if you want one, create it in the same way as the footers below
\fancyfoot[L]{} % Empty left footer
\fancyfoot[C]{} % Empty center footer
\fancyfoot[R]{\thepage} % Page numbering for right footer
\renewcommand{\headrulewidth}{0pt} % Remove header underlines
\renewcommand{\footrulewidth}{0pt} % Remove footer underlines
\setlength{\headheight}{13.6pt} % Customize the height of the header

\numberwithin{equation}{section} % Number equations within sections (i.e. 1.1, 1.2, 2.1, 2.2 instead of 1, 2, 3, 4)
\numberwithin{figure}{section} % Number figures within sections (i.e. 1.1, 1.2, 2.1, 2.2 instead of 1, 2, 3, 4)
\numberwithin{table}{section} % Number tables within sections (i.e. 1.1, 1.2, 2.1, 2.2 instead of 1, 2, 3, 4)

\setlength\parindent{0pt} % Removes all indentation from paragraphs - comment this line for an assignment with lots of text

\newcommand{\horrule}[1]{\rule{\linewidth}{#1}} % Create horizontal rule command with 1 argument of height

%	TÍTULO Y DATOS DEL ALUMNO
%----------------------------------------------------------------------------------------

\title{	
	\normalfont \normalsize 
	\textsc{\textbf{Planificación y Gestión de Proyectos Informáticos (2018-2019)} \\ Máster Profesional de Ingeniería Informática \\ Universidad de Granada} \\ [25pt] % Your university, school and/or department name(s)
	\horrule{0.5pt} \\[0.4cm] % Thin top horizontal rule
	\huge Práctica A-I \\ % The assignment title
	\horrule{2pt} \\[0.5cm] % Thick bottom horizontal rule
}

\author{Alejandro Campoy Nieves \\ Luis Gallego Quero} % Nombre y apellidos y correo
\date{\normalsize\today} % Incluye la fecha actual

\usepackage[spanish, es-tabla]{babel}
\usepackage{hyperref} % Para añadir los hiperenlaces.
\hypersetup{
	colorlinks=true,
	linkcolor=blue,
	filecolor=magenta,      
	urlcolor=cyan,
}
\usepackage{graphicx}
\usepackage{amssymb, amsmath, amsbsy}
\usepackage{mathptmx}	
\usepackage{float}
\usepackage{booktabs}					%paquete para realización de tablas profesionales
\usepackage{eurosym}
%\usepackage[table]{xcolor}
%\definecolor{lightgray}{gray}{0.9}
\usepackage{xcolor}
\usepackage{colortbl}


%----------------------------------------------------------------------------------------
% DOCUMENTO
%----------------------------------------------------------------------------------------

\begin{document}
	\maketitle % Muestra el Título
	
	\newpage %inserta un salto de página
	
	\tableofcontents % para generar el índice de contenidos
	
	\listoffigures
	
	\listoftables	
	
	\newpage	
 
\section{Lista de comprobación de objetivos de negocio}


\begin{table}[H]
	\begin{center}
		\begin{tabular}{|c|l|}
			\hline 
			 & Tareas  \\ 
			\hline \hline
			 & Me he informado de la industria del cliente y a la competencia.  \\ \hline
			 & Entiendo la estructura corporativa del cliente  \\ \hline
			 & He hecho una lista de los objetivos de negocio del cliente, empezando por el objetivo general del negocio que explica el propósito principal del proyecto de diseño de red.  \\ \hline
			 & El cliente ha identificado las operaciones críticas. \\ \hline
			 & Entiendo el criterio de éxito del cliente, y las consecuencias de los
			 fallos.  \\ \hline
			 & Entiendo el alcance del diseño del proyecto.  \\ \hline
			 & He identificado las aplicaciones de red del cliente.  \\ \hline
			 & El cliente ha explicado sus políticas de fabricantes, protocolos o
			 plataformas aceptadas.  \\ \hline
			 & El cliente ha explicado sus políticas sobre soluciones abiertas frente a
			 soluciones propietarias.  \\ \hline
			 & El cliente ha explicado sus políticas sobre autoridad distribuida para el
			 diseño de la red y la implementación.  \\ \hline
			 & Sé cuál es el presupuesto del proyecto.  \\ \hline
			 & Conozco el calendario del proyecto, incluidos la fecha de entrega final
			 y los hitos principales, y creo que es alcanzable.  \\ \hline
			 & Sé qué conocimiento técnico tienen mis clientes y el personal
			 relacionado con el proyecto.  \\ \hline
			 & He discutido sobre el plan de formación del personal con el cliente  \\ \hline
			 & Soy consciente de las políticas de oficina que puedan afectar al diseño
			 de la red.  \\ \hline
		\end{tabular}
		\caption{Lista de comprobación de objetivos de negocio.}
		\label{tabla:tabla1}
	\end{center}
\end{table}

\section{Lista de objetivos de negocio}

\begin{table}[H]
	\begin{center}
		\begin{tabular}{|l|l|l|}
			\hline 
			Objetivo de negocio & Situación actual & Comentarios \\ 
			\hline \hline
			& & \\ \hline
			& & \\ \hline
			& & \\ \hline
			& & \\ \hline		
		\end{tabular}
		\caption{Lista de objetivos de negocio}
		\label{tabla:tabla3}
	\end{center}
\end{table}

\section{Lista de comprobación del análisis de objetivos técnicos de la red}

\begin{table}[H]
	\begin{center}
		\begin{tabular}{|c|l|}
			\hline 
			& Tareas  \\ 
			\hline \hline
			& He documentado los planes del cliente para expandir durante los
			próximos dos años el número de localizaciones, usuarios y servidores.  \\ \hline
			& El cliente me ha contado los planes de migración de servidores
			departamentales a un centro de datos centralizado.  \\ \hline
			& El cliente me ha contado los planes sobre integrar los datos
			almacenados en mainframes antiguos dentro de la red de la empresa.  \\ \hline
			& El cliente me ha contado los planes sobre implementar una extranet
			para comunicarse con socios u otrs compañías.  \\ \hline
			& He documentado el objetivo de disponibilidad de la red en tiempo de y/
			o MTBF y MTTR.  \\ \hline
			& He documentado los objetivos de utilización máxima media de la red.  \\ \hline
			& He documentado los objetivos de tasa de transferencia de la red.  \\ \hline
			& He documentado los objetivos de tasa de paquetes por segundo en los
			dispositivos de interconexión de red.  \\ \hline
			& He documentado los objetivos de precisión y tasa de error aceptable.  \\ \hline
			& He discutido con el cliente la importancia de usar tramas grandes para
			maximizar la eficiencia.  \\ \hline
			& He discutido con el cliente las decisiones de compromiso asociados
			con tamaños de trama grandes, y el retardo de serialización.  \\ \hline
			& He identificado las aplicaciones que requieren un tiempo de respuesta
			más restrictivo que el estándar o inferiores a 100ms.  \\ \hline
			& He discutido con el cliente los riesgos de seguridad y los requisitos.  \\ \hline
			& He obtenido los requisitos de gestión de la red, incluyendo las metas de
			rendimiento, fallo, configuración seguridad y gestión de contabilidad.  \\ \hline
			& He actualizado el diagrama de aplicaciones de red para incluir los objetivos técnicos de las aplicaciones.  \\ \hline
			& Junto al cliente, he desarrollado una lista de objetivos de la red,
			incluyendo tanto objetivos técnicos como de negocio. La lista
			comienza con un objetivo general e incluye el resto de las metas en
			orden de prioridad. Se han marcado las metas críticas.  \\ \hline
		\end{tabular}
		\caption{Lista de comprobación del análisis de objetivos técnicos de la red.}
		\label{tabla:tabla2}
	\end{center}
\end{table}

\section{Lista de objetivos técnicos de la red}

\begin{table}[H]
	\begin{center}
		\begin{tabular}{|l|l|l|}
			\hline 
			Objetivo técnico & Importancia & Comentarios \\ 
			\hline \hline
			& & \\ \hline
			& & \\ \hline
			& & \\ \hline
			& & \\ \hline		
		\end{tabular}
		\caption{Lista de objetivos técnicos de la red.}
		\label{tabla:tabla4}
	\end{center}
\end{table}

%\newpage
%\bibliographystyle{plain}
%\bibliography{biblio}

\end{document}       
%---------------------------------------------------
