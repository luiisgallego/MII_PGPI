\input{preambuloSimple.tex}
%	TÍTULO Y DATOS DEL ALUMNO
%----------------------------------------------------------------------------------------

\title{	
	\normalfont \normalsize 
	\textsc{\textbf{Planificación y Gestión de Proyectos Informáticos (2018-2019)} \\ Máster Profesional de Ingeniería Informática \\ Universidad de Granada} \\ [25pt] % Your university, school and/or department name(s)
	\horrule{0.5pt} \\[0.4cm] % Thin top horizontal rule
	\huge Presupuesto Alternativo \\ % The assignment title
	\horrule{2pt} \\[0.5cm] % Thick bottom horizontal rule
}

\author{Alejandro Campoy Nieves \\ Luis Gallego Quero} % Nombre y apellidos y correo
\date{\normalsize\today} % Incluye la fecha actual

\usepackage[spanish, es-tabla]{babel}
\usepackage{hyperref} % Para añadir los hiperenlaces.
\hypersetup{
	colorlinks=true,
	linkcolor=blue,
	filecolor=magenta,      
	urlcolor=cyan,
}
\usepackage{graphicx}
\usepackage{amssymb, amsmath, amsbsy}
\usepackage{mathptmx}	
\usepackage{float}
\usepackage{booktabs}					%paquete para realización de tablas profesionales
\usepackage{eurosym}
%\usepackage[table]{xcolor}
%\definecolor{lightgray}{gray}{0.9}
\usepackage{xcolor}
\usepackage{colortbl}


%----------------------------------------------------------------------------------------
% DOCUMENTO
%----------------------------------------------------------------------------------------

\begin{document}
	\maketitle % Muestra el Título
	
	\newpage %inserta un salto de página
	
	\tableofcontents % para generar el índice de contenidos
	
	%\listoffigures
	
	\listoftables	
	
	\newpage	
 
\section{Descripción de la situación}

\begin{itemize}
	\item Inversión inicial realizada a través de capital de la empresa para cubrir todos los gastos.
	\item Se reciben dos ingresos, uno a mitad y otro al final del tiempo estimado de desarrollo (4 meses).	
\end{itemize}

En función de los gastos que se ven a tener, asumimos un capital inicial de 21000 \euro. La cantidad de dinero que debemos de percibir son los gastos que vamos a asumir más el sueldo del personal (considerado un gasto más). Esto hace un total de 20446,06 \euro. 

\section{flujo de caja}

\begin{table}[H]
	\begin{center}
		\begin{tabular}{|c||c|c|c|c|c|}
			\hline
			Meses & 0 & 1 & 2 & 3 & 4 \\
			\hline \hline
			Ingresos & & & 10000 & & 10000
			Personal (sueldos) &  & -3400 & -3400 & -3400 & -3400 \\ \hline \hline
			Equipo personal fijo & -328,06 &  &  &  &  \\ \hline
			Material fungible & -750 &  &  &  &  \\ \hline
			Desplazamientos y dietas & & -200 & -200 & -200 & -200 \\ \hline
			Seminario Deep Learning & -2000 &  &  &  &  \\ \hline
			Seminario sobre resonancias & -500 &  &  &  &  \\ \hline
			Varios &  & -100  & -100 & -100 &  \\ \hline
		\end{tabular}
		\caption{Tabla muy sencilla.}
		\label{tabla:sencilla}
	\end{center}
\end{table}



\newpage
\bibliographystyle{plain}
\bibliography{biblio}

\end{document}       
%---------------------------------------------------
