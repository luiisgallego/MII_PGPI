\documentclass[a4paper,12pt,oneside]{article}

\usepackage[spanish, es-tabla]{babel}
\usepackage[hidelinks]{hyperref}
\hypersetup{
    colorlinks=true,
    linkcolor=blue,
    filecolor=magenta,      
    urlcolor=cyan,
}
\usepackage{graphicx}
\usepackage{amssymb, amsmath, amsbsy}
\usepackage{mathptmx}	
\usepackage{float}
\usepackage{booktabs}					%paquete para realización de tablas profesionales
\usepackage{eurosym}
%\usepackage[table]{xcolor}
%\definecolor{lightgray}{gray}{0.9}
\usepackage{xcolor}
\usepackage{colortbl}



\title{Presupuesto I+D}
\author{Luis Gallego Quero}
\date{\today}

\begin{document}
\maketitle			
 
\begin{table}[H]
	%\begin{center}
		\begin{tabular}{|l|c|} \toprule
		\cellcolor{red} Gastos elegibles & Importe solicitado \\
		\hline \hline
		\cellcolor[gray]{0.5}GASTOS DE PERSONAL &  \\ \hline
		Total gastos de contratación de personal investigador & 0\euro \\ \hline
		Total gastos de contratación de personal experto & 15.368\euro \\ \hline
		\cellcolor[gray]{0.5}GASTOS DE EJECUCIÓN &  \\ \hline
		\cellcolor[gray]{0.8}\textbf{Costes de adquisición de material inventariable} &  \\ \hline
		Equipo para personal fijo 1 & 186,66\euro  \\ \hline
		Equipo para personal fijo 2 & 59,4\euro \\ \hline
		\cellcolor[gray]{0.8}\textbf{Costes de adquisición de material fungible} &  \\ \hline
		Material de oficina & 200\euro \\ \hline
		Pizarra y utensilios para reuniones & 100\euro \\ \hline
		Consumibles & 100\euro \\ \hline
		Mesas de oficina para personal &  200\euro \\ \hline
		Sillas de oficina para personal &  150\euro \\ \hline
		\cellcolor[gray]{0.8}\textbf{Costes de investigación contractual, conocimientos } \\
		\cellcolor[gray]{0.8}\textbf{técnicos y patentes} & \\ \hline
		Estudio realizado para elegir la mejor opción de librería\\
		Deep Learning & 500\euro   \\ \hline
		Coste librería & 0\euro \\ \hline
		\cellcolor[gray]{0.8}\textbf{Costes de consultoría, prestación de servicios, suministros,etc.} &  \\ \hline
		\cellcolor[gray]{0.8}\textbf{Costes de subcontratación} &  \\ \hline
		\cellcolor[gray]{0.5}GASTOS COMPLEMENTARIOS &  \\ \hline
		\cellcolor[gray]{0.8}Gastos de desplazamiento, viajes, estancias    \\ 
		\cellcolor[gray]{0.8}y dietas (derivados del proyecto) &  \\ \hline
		Desplazamiento a las reuniones con los clientes & 100\euro \\ \hline
		Desplazamiento a seminarios & 300\euro \\ \hline
		Dietas &  400\euro \\ \hline
		\cellcolor[gray]{0.8}Gastos de material de difusión, publicaciones, promoción,   \\ 
		\cellcolor[gray]{0.8}catálogos, folletos, cartelería, etc. &  \\ \hline
		\cellcolor[gray]{0.8}Gastos de inscripción en congresos y \\
		\cellcolor[gray]{0.8}seminarios relacionados con la actividad. &  \\ \hline
		Seminarios Deep Learning &  2000\euro \\ \hline
		Seminarios sobre resonancias & 500\euro  \\ \hline
		\cellcolor[gray]{0.8}Otros gastos de funcionamiento derivados de la actividad de \\
		i\cellcolor[gray]{0.8}nvestigación. &  \\ \hline
		Gastos varios & 300\euro \\ \hline
		\cellcolor{red}TOTAL INCENTIVO SOLICITADO & \textbf{20446,06} \euro \\ \bottomrule
		\end{tabular}
	%\end{center}
	\caption{}
	\label{nombre_etiqueta}
\end{table}

\section*{Justificación y consideraciones sobre el presupuesto}

\subsection*{Gastos de personal}

El equipo cuenta con dos ingenieros informáticos, uno especializado en desarrollo de software y otro en inteligencia artificial, lo que será mas que suficiente para la correcta culminación del proyecto. Este durará 4 meses por lo que la estimación de coste del personal, junto con la deducción del IRPF, es el siguiente: \\

2 Ingenieros Informáticos * 4 meses * (1.700\euro /mes *1.13) = \textbf{15368\euro} \\

La deducción aproximada del IRPF se ha obtenido del siguiente \href{https://infoautonomos.eleconomista.es/contratar-trabajadores/calcular-retenciones-IRPF-nominas/}{enlace}, y puesto que el sueldo anual sobrepasaría los 20.000\euro se ha tomado el 13\% de IRPF.

\subsection*{Gastos de Ejecución}

\subsubsection*{Material inventariable}

Cada uno de los trabajadores utilizará su portátil propio para la realización del trabajo. Estos fueron adquiridos tan solo hace dos meses para el nuevo curso, por tanto según el siguiente \href{https://cuentica.com/asesoria/tabla-anos-y-porcentajes-de-amortizacion-sociedades-a-partir-de-2015/}{enlace} su periodo de amortización máximo es de 10 años con una amortización del 20\% al año ya que se enmarcan en la categoría de ``equipos electrónicos". \\

El coste de uno de los portátiles fue de  2800\euro, por tanto si han pasado dos meses, esto equivale a una amortización del 0.033\% lo que deja su coste actual en 2706,67\euro. Finalmente, si el proyecto tiene una duración de cuatro meses, el coste del equipo informático es 186,66\euro.

En el segundo caso, el coste del portátil fue de 900\euro, si han pasado también dos meses tenemos la amortización anterior, lo cual deja su coste en 870,3\euro. Finalmente, con los cuatro meses de duración del proyecto, el coste del equipo es de 59,4\euro.

\subsubsection*{Material fungible}

Dentro de este apartado se han representado diferentes materiales imprescindibles par el trabajo de un proyecto, siendo su coste estándar y medio respecto a lo que podemos encontrar en el mercado.

\subsection*{Costes de investigación}

Dentro del mundo del Deep Learning existe una cantidad considerable de librerías con un gran potencial para la labor que nos ocupa. Elegir una opción viable y óptima tuvo un coste de 500\euro. Esto posteriormente ha tenido un repercusión positiva en el coste de adquirir su licencia, ya que se hará uso de Keras, la cual es gratuita.

\subsection{Gastos complementarios}

\subsubsection{Desplazamientos y seminarios}

Los clientes, como ya se han mencionado en este proyecto, se encuentra en el hospital del Parque Tecnológico de la Salud (PTS), en Granada. Los costes por desplazamiento teniendo en cuenta que el personal vive fuera resultan significativos para el proyecto. Uno vive en Lopera y otro en Marmolejo, ambos municipios de la provincia de Jaén. Además, está previsto realizar una inversión en seminarios, estos viajes también deben ser considerados (del precio de los seminarios hablamos más adelante). Las dietas son imprescindibles en en estos viajes dado que van a pasar muchas horas fuera de casa. En total se estiman alrededor de 800 \euro en estos factores.  

\subsubsection{Seminarios}

Se desea entrenar al personal en diversos seminarios con el objetivo de adquirir una mejor experiencia con el producto que se quiere desarrollar. Los seminarios son:

\begin{itemize}
	\item Seminarios de \href{https://kschool.com/cursos/programa-deep-learning/}{Deep Learning}.
	\item Seminarios sobre \href{https://www.emagister.com/cursos-resonancia-magnetica-kwes-11454.htm}{resonancias}.
\end{itemize}

Es fundamental adquirir algo de conocimiento experto sobre las resonancias para saber que datos son importantes y como den ser analizados. El conocimiento del Deep Learning para poder aplicar este conocimiento en entrenar un modelo es importante en una magnitud similar. Por ello estimamos que estos seminarios nos van a suponer un coste de 2500 \euro teniendo en cuenta algunos seminarios que se ofertan en las referencias.

\subsubsection{Varios}

Hemos realizado una estimación de gastos que no hemos previsto por algún motivo, dejando un margen de error de 300 \euro por si ocurre algún tipo de emergencia o imprevisto.


\end{document}       
%---------------------------------------------------
