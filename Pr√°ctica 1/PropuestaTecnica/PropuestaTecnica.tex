\documentclass[a4paper,12pt,oneside]{article}

\usepackage[spanish, es-tabla]{babel}
\usepackage[hidelinks]{hyperref}
\usepackage{graphicx}
\usepackage{amssymb, amsmath, amsbsy}
\usepackage{mathptmx}	
\usepackage{float}

\title{Aplicación de Detección Temprana de Frácturas mediante Resonancias}
\author{Luis Gallego Quero \\ Alejandro Campoy Nieves}

\date{\today}

\begin{document}
\maketitle			
 
\section{Resumen}

Nuestro proyecto trata sobre el desarrollo de una aplicación capaz de analizar un gran almacén de imágenes de resonancias magnéticas mediante Deep Learning, lo que nos proporcionará un reconocimiento de fracturas sin la necesidad de la supervisión de un profesional. 

Además, nuestro sistema será capaz de detectar indicios de futuras fracturas, siendo estas aún pequeñas fisuras o esguinces.

\section{Lugar de Ejecución}

Inicialmente la propuesta está enfocada al Parque Tecnológico de la Salud (PTS) de Granada. Con unas posibilidades de expansión para futuros hospitales que requieran este servicio.

\section{Objetivos}
\subsection{Generales} 
\begin{itemize}
	\item Mejorar el desempeño de los profesionales.
	\item Ahorro de tiempo en el diagnóstico.
	\item Incrementar la tasa de acierto en el diagnóstico.
	\item Introducir técnicas de Deep Learning en los hospitales.
	\item Mejora de la calidad sanitaria.
\end{itemize}

\subsection{Específicos}
\begin{itemize}
	\item Detección precoz de fracturas, siendo estas aún pequeñas fisuras o esguinces.
	\item Tasa de fallo inferior al 1\% en los falsos negativos.
	\item Eficiencia en el proceso de análisis, siendo posible realizar diagnósticos en pocos segundos.
\end{itemize}

\section{Antecedentes}

El grupo está formado por estudiantes de la Escuela Técnica Superior de Ingeniería Informática y Telecomunicaciones con estudios de ingeniero técnico informático. Por ello, la única experiencia actual reside en proyectos anteriores realizados que están relacionados con el proyecto en cuestión. Por ejemplo, la implementación de algunas estrategias de inteligencia artificial para solucionar distintos problemas como podría ser la clasificación de ciertos objetos y prácticas relacionadas con la visión por computador.

\section{Justificación}

Hemos realizado una búsqueda de trabajos similares o estrechamente relacionados que puedan ser relevantes para este proyecto. Entre los resultados obtenidos, hemos encontrado algunas aplicaciones de interés como un algoritmo de detección de \href{https://blogthinkbig.com/este-algoritmo-diagnostica-neumonia-con-la-precision-de-un-medico}{neumonías} o una \href{https://www.face2gene.com/}{aplicación} que realiza evaluaciones genéticas completas y precisas a partir de un reconocimiento facial. \\

También como documentación interesante hay que tener en cuenta la viabilidad del proyecto y obtener conocimiento sobre el problema que se pretende resolver mediante Deep Learning. Este \href{https://blog.hospitalsanangelinn.mx/resonancia-magnetica-diagnostico}{enlace} nos ofrece información de qué cosas se pueden detectar mediante una resonancia magnética teniendo conocimiento suficiente para poder interpretarla. \\

Un ejemplo de fractura por tensión perfectamente identificado sería el que aparece en la \href{https://www.google.es/search?q=resonancia+magnetica+con+fractura&source=lnms&tbm=isch&sa=X&ved=0ahUKEwj_ydaisK7eAhVDgRoKHeGoDsUQ_AUIDigB&biw=1536&bih=754#imgrc=5wcLStUv22LjAM:}{imagen de referencia}.

\section{Innovación}

Basándonos en una de las nuevas tecnologías emergentes, que ha pasado de ser una moda a una realidad capaz de solucionar problemas reales, siendo esta Deep Learning. Utilizando esta técnica, queremos agilizar los diagnósticos médicos proporcionando directamente un resultado de la resonancia sin necesidad de llevar a cabo un estudio por parte de un profesional.

\section{Actividades a realizar alineadas con los objetivos}

Hemos pensado en una serie de actividades a realizar con la finalidad de obtener los objetivos previstos:

\begin{itemize}
	\item Estudio del problema. Para obtener conocimiento experto.
	\item Análisis de parámetros de entrada a tener en cuenta.
	\item Desarrollo de un algoritmo de aprendizaje Deep Learning.
	\item Entrenamiento de la red neuronal con ejemplos de resonancias.
	\item Prueba del modelo entrenado con resonancias exteriores al conjunto de entrenamiento.
\end{itemize}

\section{Cronograma}

\section{Cauces de seguimiento}

\section{Valor añadido}

Dentro de los avances que supone el proyecto que abarcamos, podemos encontrar una mejora de eficiencia y calidad en el diagnóstico. De esta forma, ganamos en seguridad de los resultados obtenidos al mismo tiempo que mejoramos la rapidez en la que se atiende a pacientes con este tipo de problemas de salud.

\section{Beneficios y beneficiarios}

En este proyecto distinguimos dos beneficiarios principalmente, con una serie de beneficios distintos aunque similares para cada uno. \\

Por un lado, los pacientes, como ya se ha mencionado anteriormente, consiguen una mejora en la calidad de los diagnósticos que se les realiza, incrementando la tasa de acierto al mismo tiempo que no tienen la necesidad de esperar a ser evaluados por un experto cualificado. \\

Por otro lado, los hospitales que implementaran este sistema, verían una mejora económica debido a la minimización de recursos necesarios y tiempo. De tal forma que tendrían a esos mismos pacientes con una mejor calidad de servicio.






%%%%%% RECURSOS %%%%%%




% \textbf{negrita}
%  \textit{cursiva}

% Insertar imagen
%\begin{figure}[h]
	%\centering
		%\includegraphics[scale=0.1]{./ruta}
	%\caption{Texto parte inferior}
	%\label{etiqueta para referencia}
%\end{figure}

% 		LISTADO DE ITEMS
%\begin{itemize}
	%\item 
%\end{itemize}

% 		BIBLIOGRAFÍA
%\begin{thebibliography}{X}
	%\bibitem{nombreItem} \url{enlace} 
%\end{thebibliography}


\end{document}       
%---------------------------------------------------
