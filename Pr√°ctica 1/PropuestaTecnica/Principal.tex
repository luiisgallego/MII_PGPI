\documentclass[a4paper,12pt,oneside]{article}

\usepackage[spanish, es-tabla]{babel}
\usepackage[hidelinks]{hyperref}
\usepackage{graphicx}
\usepackage{amssymb, amsmath, amsbsy}
\usepackage{mathptmx}	
\usepackage{float}

\title{Aplicación de Detección Temprana de Frácturas mediante Resonancias}
\author{Luis Gallego Quero \\ Alejandro Campoy Nieves}

\date{\today}

\begin{document}
\maketitle			
 
\section{Resumen}

Nuestro proyecto trata sobre el desarrollo de una aplicación capaz de analizar un gran almacén de imágenes de resonancias magnéticas mediante Deep Learning, lo que nos proporcionará un reconocimiento de fracturas sin la necesidad de la supervisión de un profesional. 

Además, nuestro sistema será capaz de detectar indicios de futuras fracturas, siendo estas aún pequeñas fisuras o esguinces.

\section{Lugar de Ejecución}

Inicialmente la propuesta está enfocada al Parque Tecnológico de la Salud (PTS) de Granada. Con unas posibilidades de expansión para futuros hospitales que requieran este servicio.

\section{Objetivos}
\subsection{Generales} 
\begin{itemize}
	\item Mejorar el desempeño de los profesionales.
	\item Ahorro de tiempo en el diagnóstico.
	\item Incrementar la tasa de acierto en el diagnóstico.
	\item Introducir técnicas de Deep Learning en los hospitales.
	\item Mejora de la calidad sanitaria.
\end{itemize}

\subsection{Específicos}
\begin{itemize}
	\item Detección precoz de fracturas, siendo estas aún pequeñas fisuras o esguinces.
	\item Tasa de fallo inferior al 1\% en los falsos negativos.
	\item Eficiencia en el proceso de análisis, siendo posible realizar diagnósticos en pocos segundos.
\end{itemize}



%%%%%% RECURSOS %%%%%%




% \textbf{negrita}
%  \textit{cursiva}

% Insertar imagen
%\begin{figure}[h]
	%\centering
		%\includegraphics[scale=0.1]{./ruta}
	%\caption{Texto parte inferior}
	%\label{etiqueta para referencia}
%\end{figure}

% 		LISTADO DE ITEMS
%\begin{itemize}
	%\item 
%\end{itemize}

% 		BIBLIOGRAFÍA
%\begin{thebibliography}{X}
	%\bibitem{nombreItem} \url{enlace} 
%\end{thebibliography}


\end{document}       
%---------------------------------------------------
