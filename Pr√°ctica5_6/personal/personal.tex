%%%%%%%%%%%%%%%%%%%%%%%%%%%%%%%%%%%%%%%%%
% Short Sectioned Assignment LaTeX Template Version 1.0 (5/5/12)
% This template has been downloaded from: http://www.LaTeXTemplates.com
% Original author:  Frits Wenneker (http://www.howtotex.com)
% License: CC BY-NC-SA 3.0 (http://creativecommons.org/licenses/by-nc-sa/3.0/)
%%%%%%%%%%%%%%%%%%%%%%%%%%%%%%%%%%%%%%%%%

%----------------------------------------------------------------------------------------
%	PACKAGES AND OTHER DOCUMENT CONFIGURATIONS
%----------------------------------------------------------------------------------------

\documentclass[paper=a4, fontsize=11pt]{scrartcl} % A4 paper and 11pt font size

% ---- Entrada y salida de texto -----

\usepackage[T1]{fontenc} % Use 8-bit encoding that has 256 glyphs
\usepackage[utf8]{inputenc}
%\usepackage{fourier} % Use the Adobe Utopia font for the document - comment this line to return to the LaTeX default

% ---- Idioma --------

\usepackage[spanish, es-tabla]{babel} % Selecciona el español para palabras introducidas automáticamente, p.ej. "septiembre" en la fecha y especifica que se use la palabra Tabla en vez de Cuadro

% ---- Otros paquetes ----

\usepackage{url} % ,href} %para incluir URLs e hipervínculos dentro del texto (aunque hay que instalar href)
\usepackage{amsmath,amsfonts,amsthm} % Math packages
%\usepackage{graphics,graphicx, floatrow} %para incluir imágenes y notas en las imágenes
\usepackage{graphics,graphicx, float} %para incluir imágenes y colocarlas

% Para hacer tablas comlejas
%\usepackage{multirow}
%\usepackage{threeparttable}

%\usepackage{sectsty} % Allows customizing section commands
%\allsectionsfont{\centering \normalfont\scshape} % Make all sections centered, the default font and small caps

\usepackage{fancyhdr} % Custom headers and footers
\pagestyle{fancyplain} % Makes all pages in the document conform to the custom headers and footers
\fancyhead{} % No page header - if you want one, create it in the same way as the footers below
\fancyfoot[L]{} % Empty left footer
\fancyfoot[C]{} % Empty center footer
\fancyfoot[R]{\thepage} % Page numbering for right footer
\renewcommand{\headrulewidth}{0pt} % Remove header underlines
\renewcommand{\footrulewidth}{0pt} % Remove footer underlines
\setlength{\headheight}{13.6pt} % Customize the height of the header

\numberwithin{equation}{section} % Number equations within sections (i.e. 1.1, 1.2, 2.1, 2.2 instead of 1, 2, 3, 4)
\numberwithin{figure}{section} % Number figures within sections (i.e. 1.1, 1.2, 2.1, 2.2 instead of 1, 2, 3, 4)
\numberwithin{table}{section} % Number tables within sections (i.e. 1.1, 1.2, 2.1, 2.2 instead of 1, 2, 3, 4)

\setlength\parindent{0pt} % Removes all indentation from paragraphs - comment this line for an assignment with lots of text

\newcommand{\horrule}[1]{\rule{\linewidth}{#1}} % Create horizontal rule command with 1 argument of height

%	TÍTULO Y DATOS DEL ALUMNO
%----------------------------------------------------------------------------------------

\title{	
	\normalfont \normalsize 
	\textsc{\textbf{Planificación y Gestión de Proyectos Informáticos (2018-2019)} \\ Máster Profesional de Ingeniería Informática \\ Universidad de Granada} \\ [25pt] % Your university, school and/or department name(s)
	\horrule{0.5pt} \\[0.4cm] % Thin top horizontal rule
	\huge Criterios de Selección de Personal \\ % The assignment title
	\horrule{2pt} \\[0.5cm] % Thick bottom horizontal rule
}

\author{Alejandro Campoy Nieves \\ Luis Gallego Quero} % Nombre y apellidos y correo
\date{\normalsize\today} % Incluye la fecha actual

\usepackage[spanish, es-tabla]{babel}
\usepackage{hyperref} % Para añadir los hiperenlaces.
\hypersetup{
	colorlinks=true,
	linkcolor=blue,
	filecolor=magenta,      
	urlcolor=blue,
}
\usepackage{graphicx}
\usepackage{amssymb, amsmath, amsbsy}
\usepackage{mathptmx}	
\usepackage{float}
\usepackage{booktabs}					%paquete para realización de tablas profesionales
\usepackage{eurosym}

\usepackage[table]{xcolor}
\usepackage{color}
\usepackage{colortbl}
\usepackage{multicol}
\usepackage{multirow}
\usepackage{booktabs}
\usepackage{tabularx}
%---- Paquete para aliniación de texto verticalmente dentro de tablas
\usepackage{array}
%---- Paquetes para los pies de las tablas
\usepackage{caption}
\usepackage{subcaption}


%----------------------------------------------------------------------------------------
% DOCUMENTO
%----------------------------------------------------------------------------------------

\begin{document}
	\maketitle % Muestra el Título
	
	\newpage %inserta un salto de página
	
	\tableofcontents % para generar el índice de contenidos
	
	%\listoffigures
	
	\listoftables	
	
	\newpage	
 
\section{Oferta de Trabajo}

Si te interesa trabajar en un equipo de desarrollo con una gran capacidad de evolución y aportación a la ciencia, esta es tu oportunidad. Como pequeña empresa, somos un startup formada provisionalmente por dos integrantes con la motivación de desarrollar un proyecto sobre diagnósticos automáticos de fracturas a partir de resonancias magnéticas. \\

Buscamos a un ingeniero informático, con buena capacidad de programación y conocimiento en Deep Learning. Los requisitos fundamentales que se buscan son:

\begin{itemize}
	\item Estudios mínimos: grado en ingeniería informática.
	\item Experiencia mínima: No se requiere experiencia laboral previa.	
	\item Conocimientos necesarios:
		\begin{itemize}
			\item Python.
			\item Programación orientada a objetos.
			\item Conocimiento en el uso de APIs.
			\item Desarrollo de interfaces.
			\item Git y GitHub.
			\item Metodologías Ágiles de desarrollo.
			\item Tecnologías de la información.
		\end{itemize}
	\item Requisitos mínimos:
	\begin{itemize}
		\item Experiencia en Deep Learning muy recomendable.
		\item Experiencia en análisis de requisitos funcionales y no funcionales.
		\item Experiencia con herramientas de trabajo colaborativo.
		\item Disponibilidad inmediata.
		\item Inglés nivel B1.
	\end{itemize}
\end{itemize}

Se oferta:

\begin{itemize}
	\item Puestos de trabajo: 1.
	\item Salario: 1.700 \euro mensuales, 20.400 \euro anuales.
	\item Jornada laboral: 35 horas a la semana, desde 8:00 hasta 15:00 de Lunes a Viernes.
\end{itemize}

\newpage

\section{Preselección de Candidatos}

Para realizar una buena preselección de candidatos, se han clasificado los requisitos con valores dentro del intervalo 0 y 5 ambos incluidos. De tal manera que definimos el grado de relevancia que supone para nosotros cumplir estos factores. Posteriormente, todo candidato que supere el 3 sobre 5 de media será llamado para una entrevista personal.

\begin{table}[H]
	\begin{center}
		\begin{tabular}{|l||c|}
			\hline 
			Criterio & Valoración \\
			\hline \hline
			Python & 3 \\ \hline
			Programación orientada a objetos & 2 \\ \hline
			Conocimiento en el uso de APIs & 2 \\ \hline
			Desarrollo de interfaces & 1 \\ \hline
			Git y GitHub & 2 \\ \hline
			Metodologías Ágiles de desarrollo & 3 \\ \hline
			Tecnologías de la información & 2 \\ \hline
			Experiencia en Deep Learning & 5 \\ \hline
			Experiencia en análisis de requisitos funcionales y no funcionales & 3 \\ \hline
			Experiencia con herramientas de trabajo colaborativo & 2 \\ \hline
			Disponibilidad inmediata & 4 \\ \hline
			Inglés nivel B1 & 4 \\ \hline
		\end{tabular}
		\caption{Relevancia de requisitos para preselección de candidatos.}
		\label{tabla:preseleccion}
	\end{center}
\end{table}

\section{Entrevistas con los candidatos}

En primer lugar se realizará una entrevista previa al candidato. De esta forma se obtiene la posibilidad, no solo de saber su experiencia básica, sino de observar su forma de ser y la forma de reaccionar que tiene. \\

La idea fundamental es el desarrollo de una ficha de tal forma que pueda ser rellenada por la persona que entreviste a esa persona, para poder recopilar esta información de una forma estandarizada y organizada.


\begin{table}[H]
	\begin{center}
		\begin{tabular}{|p{4cm}|p{10cm}|}
			\hline 
			\multicolumn{2}{|c|}{\textbf{Ficha entrevista}} \\
			\hline
			Experiencia laboral & \parbox[l][0.3\textwidth][c]{8cm}{
				\begin{itemize}
					\item Sin experiencia.
					\item Inferior a 2 años.
					\item Entre 2 y 4 años.
					\item Superior a 4 años.
			\end{itemize} } \\ \hline
			Estudios realizados & \parbox[l][0.4\textwidth][c]{8cm}{
				\begin{itemize}
					\item Ingeniería informática.
					\item Otra carrera relacionada.
					\item Grado superior.
					\item Grado medio.
					\item Otros estudios.
			\end{itemize} } \\ \hline
			Referencia personal & \parbox[l][0.3\textwidth][c]{8cm}{
				\begin{itemize}
					\item Trabajos individuales.
					\item Trabajos en equipo.
					\item Aportaciones software libre.
					\item Etapa en el extranjero.
			\end{itemize} } \\ \hline
		Capacidades & \parbox[l][0.4\textwidth][c]{8cm}{
			\begin{itemize}
				\item Desarrollo de algoritmos de Deep Learning.
				\item Conocimiento en el uso de APIs.
				\item Desarrollo de interfaces.
				\item Tecnologías de la información.
				\item Idiomas.
				\item Otros.
		\end{itemize} } \\ \hline
		\end{tabular}
	\end{center}
\end{table}
\newpage
\begin{table}[H]
	\begin{center}
		\begin{tabular}{|p{4cm}|p{10cm}|}
			\hline 
				Motivación & \parbox[l][0.6\textwidth][c]{8cm}{
				\begin{itemize}
					\item ¿Por qué busca este trabajo?
					\begin{itemize}
						\item Necesidad de trabajo.
						\item Necesidad de experiencia.
						\item Formación en el área.
						\item Interés personal.
						\item Aporte a la sanidad.
						\item Otras.
					\end{itemize}
					\item ¿Perspectivas de futuro?
					\begin{itemize}
						\item Mejorar su vida laboral.
						\item Nuevos proyectos.
						\item Mejora de rango en la empresa.
						\item Otros.
					\end{itemize}
				\end{itemize} } \\ \hline
				Cuestiones varias & \parbox[l][0.2\textwidth][c]{8cm}{
					\begin{itemize}
						\item Salario.
						\item Disponibilidad.
						\item Amoldación a jornada.
				\end{itemize} } \\ \hline
				Cuestiones & \parbox[l][0.2\textwidth][c]{8cm}{
									 } \\ \hline
		\end{tabular}
		\caption{Ficha para entrevistas a candidatos}
		\label{tabla:ficha}
	\end{center}
\end{table}

\newpage

\begin{table}[H]
	\begin{center}
		\begin{tabular}{|p{3.5cm}|p{8cm}|p{3.5cm}|}
			\hline 
			\multicolumn{3}{|c|}{\textbf{Evaluación}} \\
			\hline
			Experiencia laboral & \parbox[l][0.3\textwidth][c]{8cm}{
				\begin{itemize}
					\item Sin experiencia.
					\item Inferior a 2 años.
					\item Entre 2 y 4 años.
					\item Superior a 4 años.
			\end{itemize} } & \\ \hline
			Estudios realizados & \parbox[l][0.4\textwidth][c]{8cm}{
				\begin{itemize}
					\item Ingeniería informática.
					\item Otra carrera relacionada.
					\item Grado superior.
					\item Grado medio.
					\item Otros estudios.
			\end{itemize} } & \\ \hline
			Referencia personal & \parbox[l][0.3\textwidth][c]{8cm}{
				\begin{itemize}
					\item Trabajos individuales.
					\item Trabajos en equipo.
					\item Aportaciones software libre.
					\item Etapa en el extranjero.
			\end{itemize} } & \\ \hline
			Capacidades & \parbox[l][0.4\textwidth][c]{8cm}{
				\begin{itemize}
					\item Desarrollo de algoritmos de Deep Learning.
					\item Conocimiento en el uso de APIs.
					\item Desarrollo de interfaces.
					\item Tecnologías de la información.
					\item Idiomas.
					\item Otros.
			\end{itemize} } & \\ \hline
		\end{tabular}
	\end{center}
\end{table}
\newpage
\begin{table}[H]
	\begin{center}
		\begin{tabular}{|p{3.5cm}|p{8cm}|p{3.5cm}|}
			\hline 
			Motivación & \parbox[l][0.6\textwidth][c]{8cm}{
				\begin{itemize}
					\item ¿Por qué busca este trabajo?
					\begin{itemize}
						\item Necesidad de trabajo.
						\item Necesidad de experiencia.
						\item Formación en el área.
						\item Interés personal.
						\item Aporte a la sanidad.
						\item Otras.
					\end{itemize}
					\item ¿Perspectivas de futuro?
					\begin{itemize}
						\item Mejorar su vida laboral.
						\item Nuevos proyectos.
						\item Mejora de rango en la empresa.
						\item Otros.
					\end{itemize}
			\end{itemize} } & \\ \hline
			Cuestiones varias & \parbox[l][0.2\textwidth][c]{8cm}{
				\begin{itemize}
					\item Salario.
					\item Disponibilidad.
					\item Amoldación a jornada.
			\end{itemize} } & \\ \hline
			Cuestiones & \parbox[l][0.2\textwidth][c]{8cm}{
			} & \\ \hline
		\end{tabular}
		\caption{Evaluación de fichas sobre entrevistas a candidatos.}
		\label{tabla:evaluación}
	\end{center}
\end{table}
\newpage
\bibliographystyle{plain}
\bibliography{biblio}

\end{document}       
%---------------------------------------------------
