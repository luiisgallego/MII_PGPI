\documentclass[a4paper,12pt,oneside]{article}

\usepackage[spanish, es-tabla]{babel}
\usepackage[hidelinks]{hyperref}
\hypersetup{
    colorlinks=true,
    linkcolor=blue,
    filecolor=magenta,      
    urlcolor=cyan,
}
\usepackage{graphicx}
\usepackage{amssymb, amsmath, amsbsy}
\usepackage{mathptmx}	
\usepackage{float}
\usepackage{booktabs}					%paquete para realización de tablas profesionales

\title{Riesgos Luis}
\author{Luis Gallego Quero}
\date{\today}

\begin{document}
\maketitle			
 
\section*{Interfaz de sanidad anticuada o difícil de tratar.}

\begin{itemize}
	\item \textbf{Identificador del riesgo: R6}
	\item \textbf{Descripción(2-3 líneas): } Llegado el momento de implantar nuestro software en el sistema informático del hospital podemos encontrarnos con un software tan precario o anticuado, que la dificultad para instalar nuestros avances sean de gran dificultad.
	\item \textbf{Probabilidad de ocurrencia: } 30\%
	\item \textbf{Impacto del riesgo: } Medio.
	\item \textbf{Monitorización del riesgo: } Realizar un pequeño estudio del software sanitario al comienzo del proyecto. 
	\item \textbf{Estrategia de mitigación del riesgo (prevención): } Utilizar software que sea fácilmente adaptable y compatible. 
	\item \textbf{Plan de contingencia (actuación en caso de que se materialice): } Construcción de API que adapte nuestro software al del hospital.
	\item \textbf{Recursos necesarios: } Ninguno?
\end{itemize}

\section*{Mala planificación o planteamiento del proyecto.}
\begin{itemize}
	\item \textbf{Identificador del riesgo: R7}
	\item \textbf{Descripción: } En el transcurso del proyecto puede darse que inicialmente hayamos planteado mal el reparto del tiempo general a los diferentes módulos o hayamos otorgado más importancia a partes que no las tenían.
	\item \textbf{Probabilidad de ocurrencia: } 40\%
	\item \textbf{Impacto del riesgo: } Alto.
	\item \textbf{Monitorización del riesgo: } Controlar el tiempo utilizado para los diferentes módulos y realizar ajustes si estos comienzan a retrasarse.
	\item \textbf{Estrategia de mitigación del riesgo (prevención): } Dedicar mas tiempo a la planificación inicial dividiendo los módulos en subtareas más concretas, lo que nos permitiría ser más precisos en la estimación.
	\item \textbf{Plan de contingencia: } Reajustar el proyecto, y en caso extremo, negociar con el cliente una prorroga.
	\item \textbf{Recursos necesarios: }
\end{itemize}

\section*{Pérdida de datos inesperada.}
\begin{itemize}
	\item \textbf{Identificador del riesgo: R8}
	\item \textbf{Descripción(2-3 líneas): } Durante el transcurso del proyecto puede ocurrir alguna incidencia en algún gestor utilizado o nuestras máquinas locales y perder parte del desarrollo del proyecto.
	\item \textbf{Probabilidad de ocurrencia: } 10\%
	\item \textbf{Impacto del riesgo: } Alto
	\item \textbf{Monitorización del riesgo: } 
	\item \textbf{Estrategia de mitigación del riesgo: } Realizar copias de seguridad constantes.
	\item \textbf{Plan de contingencia: } Desarrollar la parte perdida en el menor tiempo posible????
	\item \textbf{Recursos necesarios: }
\end{itemize}

\section*{Dificultad de coordinación y comunicación en el equipo.}
\begin{itemize}
	\item \textbf{Identificador del riesgo: R9}
	\item \textbf{Descripción: } Dificultad para coincidir en las reuniones  o poca comunicación a lo largo del transcurso del desarrollo.
	\item \textbf{Probabilidad de ocurrencia: } 20\%
	\item \textbf{Impacto del riesgo: } Bajo
	\item \textbf{Monitorización del riesgo: } Avisar unos días antes de la reunión planificada.
	\item \textbf{Estrategia de mitigación del riesgo (prevención): } Ser conscientes de la importancia de una comunicación constante, tanto de los buenos sucesos como los malos a lo largo del proyecto.
	\item \textbf{Plan de contingencia (actuación en caso de que se materialice): } Videollamadas.
	\item \textbf{Recursos necesarios: } Ninguno.
\end{itemize}

\section*{Retrasos generales acorde a la planificación establecida.}
\begin{itemize}
	\item \textbf{Identificador del riesgo: R10}
	\item \textbf{Descripción(2-3 líneas): } Algunas de las diferentes tareas que componen el proyecto comienzan a retrasarse debido a imprevistos inesperados.
	\item \textbf{Probabilidad de ocurrencia: } 60\%
	\item \textbf{Impacto del riesgo: } Alto
	\item \textbf{Monitorización del riesgo: } Control sobre el tiempo empleado en las diferentes tareas.
	\item \textbf{Estrategia de mitigación del riesgo (prevención): } Realizar una planificación más exhaustiva de las tareas a implementar.
	\item \textbf{Plan de contingencia (actuación en caso de que se materialice): } Reajuste del tiempo respecto a otras tareas que hayan ocupado menos tiempo del estimado.
	\item \textbf{Recursos necesarios: } Ninguno.
\end{itemize} 




%%%%%% RECURSOS %%%%%%

% ENÑACE
 %\href{link}{texto} 

% TABLA
%\begin{table}[H]
	%\begin{center}
		%\begin{tabular}{|c|c|} \toprule
		%1 & 2 \\ \midrule
		%2 & 4 \\
		%5 & 6\\ \bottomrule
		%\end{tabular}
	%\end{center}
%\caption{}
%\label{nombre_etiqueta}
%\end{table}

% \textbf{negrita}
%  \textit{cursiva}

% Insertar imagen
%\begin{figure}[h]
	%\centering
		%\includegraphics[scale=0.1]{./ruta}
	%\caption{Texto parte inferior}
	%\label{etiqueta para referencia}
%\end{figure}

% 		LISTADO DE ITEMS
%\begin{itemize}
	%\item 
%\end{itemize}

% 		BIBLIOGRAFÍA
%\begin{thebibliography}{X}
	%\bibitem{nombreItem} \url{enlace} 
%\end{thebibliography}

%%%%%% FIN RECURSOS %%%%%%


\end{document}       
%---------------------------------------------------
