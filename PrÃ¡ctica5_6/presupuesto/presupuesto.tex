%%%%%%%%%%%%%%%%%%%%%%%%%%%%%%%%%%%%%%%%%
% Short Sectioned Assignment LaTeX Template Version 1.0 (5/5/12)
% This template has been downloaded from: http://www.LaTeXTemplates.com
% Original author:  Frits Wenneker (http://www.howtotex.com)
% License: CC BY-NC-SA 3.0 (http://creativecommons.org/licenses/by-nc-sa/3.0/)
%%%%%%%%%%%%%%%%%%%%%%%%%%%%%%%%%%%%%%%%%

%----------------------------------------------------------------------------------------
%	PACKAGES AND OTHER DOCUMENT CONFIGURATIONS
%----------------------------------------------------------------------------------------

\documentclass[paper=a4, fontsize=11pt]{scrartcl} % A4 paper and 11pt font size

% ---- Entrada y salida de texto -----

\usepackage[T1]{fontenc} % Use 8-bit encoding that has 256 glyphs
\usepackage[utf8]{inputenc}
%\usepackage{fourier} % Use the Adobe Utopia font for the document - comment this line to return to the LaTeX default

% ---- Idioma --------

\usepackage[spanish, es-tabla]{babel} % Selecciona el español para palabras introducidas automáticamente, p.ej. "septiembre" en la fecha y especifica que se use la palabra Tabla en vez de Cuadro

% ---- Otros paquetes ----

\usepackage{url} % ,href} %para incluir URLs e hipervínculos dentro del texto (aunque hay que instalar href)
\usepackage{amsmath,amsfonts,amsthm} % Math packages
%\usepackage{graphics,graphicx, floatrow} %para incluir imágenes y notas en las imágenes
\usepackage{graphics,graphicx, float} %para incluir imágenes y colocarlas

% Para hacer tablas comlejas
%\usepackage{multirow}
%\usepackage{threeparttable}

%\usepackage{sectsty} % Allows customizing section commands
%\allsectionsfont{\centering \normalfont\scshape} % Make all sections centered, the default font and small caps

\usepackage{fancyhdr} % Custom headers and footers
\pagestyle{fancyplain} % Makes all pages in the document conform to the custom headers and footers
\fancyhead{} % No page header - if you want one, create it in the same way as the footers below
\fancyfoot[L]{} % Empty left footer
\fancyfoot[C]{} % Empty center footer
\fancyfoot[R]{\thepage} % Page numbering for right footer
\renewcommand{\headrulewidth}{0pt} % Remove header underlines
\renewcommand{\footrulewidth}{0pt} % Remove footer underlines
\setlength{\headheight}{13.6pt} % Customize the height of the header

\numberwithin{equation}{section} % Number equations within sections (i.e. 1.1, 1.2, 2.1, 2.2 instead of 1, 2, 3, 4)
\numberwithin{figure}{section} % Number figures within sections (i.e. 1.1, 1.2, 2.1, 2.2 instead of 1, 2, 3, 4)
\numberwithin{table}{section} % Number tables within sections (i.e. 1.1, 1.2, 2.1, 2.2 instead of 1, 2, 3, 4)

\setlength\parindent{0pt} % Removes all indentation from paragraphs - comment this line for an assignment with lots of text

\newcommand{\horrule}[1]{\rule{\linewidth}{#1}} % Create horizontal rule command with 1 argument of height

%	TÍTULO Y DATOS DEL ALUMNO
%----------------------------------------------------------------------------------------

\title{	
	\normalfont \normalsize 
	\textsc{\textbf{Planificación y Gestión de Proyectos Informáticos (2018-2019)} \\ Máster Profesional de Ingeniería Informática \\ Universidad de Granada} \\ [25pt] % Your university, school and/or department name(s)
	\horrule{0.5pt} \\[0.4cm] % Thin top horizontal rule
	\huge Presupuesto Alternativo \\ % The assignment title
	\horrule{2pt} \\[0.5cm] % Thick bottom horizontal rule
}

\author{Alejandro Campoy Nieves \\ Luis Gallego Quero} % Nombre y apellidos y correo
\date{\normalsize\today} % Incluye la fecha actual

\usepackage[spanish, es-tabla]{babel}
\usepackage{hyperref} % Para añadir los hiperenlaces.
\hypersetup{
	colorlinks=true,
	linkcolor=blue,
	filecolor=magenta,      
	urlcolor=cyan,
}
\usepackage{graphicx}
\usepackage{amssymb, amsmath, amsbsy}
\usepackage{mathptmx}	
\usepackage{float}
\usepackage{booktabs}					%paquete para realización de tablas profesionales
\usepackage{eurosym}
%\usepackage[table]{xcolor}
%\definecolor{lightgray}{gray}{0.9}
\usepackage{xcolor}
\usepackage{colortbl}


%----------------------------------------------------------------------------------------
% DOCUMENTO
%----------------------------------------------------------------------------------------

\begin{document}
	\maketitle % Muestra el Título
	
	\newpage %inserta un salto de página
	
	\tableofcontents % para generar el índice de contenidos
	
	%\listoffigures
	
	\listoftables	
	
	\newpage	
 
\section{Descripción de la situación}

\begin{itemize}
	\item Inversión inicial realizada a través de capital de la empresa para cubrir todos los gastos.
	\item Se reciben dos ingresos, uno a mitad y otro al final del tiempo estimado de desarrollo (4 meses).	
\end{itemize}

 Asumimos un capital inicial de la empresa para sufragar todos los gastos. Consideramos que vamos a recibir una cantidad total en ingresos por parte del cliente equivalente a 40.000\euro, pagados en dos plazos; a mitad y al final del proyecto.  

\section{flujo de caja}

\begin{table}[H]
	\begin{center}
		\begin{tabular}{|c||c|c|c|c|c|}
			\hline
			Meses & 0 & 1 & 2 & 3 & 4 \\
			\hline \hline
			Ingresos & & & 20.000 & & 20.000 \\ \hline
			Personal (sueldos) &  & -3.400 & -3.400 & -3.400 & -3.400 \\ \hline 
			Seguridad Social &  & -962,2 & -962,2 & -962,2 & -962,2 \\ \hline
			IRPF & & -442 & -442 & -442 & -442 \\ \hline
			Fondo Garantía Salarial & & -6,8 & -6,8 & -6,8 & -6,8 \\ \hline
			Seguro de desempleo & & -239,27 & -239,27 & -239,27 & -239,27 \\ \hline
			Formación Profesional & & -23,8 & -23,8 & -23,8 & -23,8 \\ \hline
			\hline
			Equipo personal fijo & -328,06 &  &  &  &  \\ \hline
			Material fungible & -750 &  &  &  &  \\ \hline
			Desplazamientos y dietas & & -200 & -200 & -200 & -200 \\ \hline
			Seminario Deep Learning & -2.000 &  &  &  &  \\ \hline
			Seminario sobre resonancias & -500 &  &  &  &  \\ \hline
			Varios &  & -100  & -100 & -100 &  \\ \hline
			\hline
			\textbf{Ingresos - Gastos} & -3.578,06 & -5.374,07 & 14.625,96 & -5.374,07 & 14.725,93 \\ \hline
			\textbf{Flujo de caja} & -3.578,06 & -8.952,13 & 5.673,83 & 299,76 & 15.025,69 \\ \hline
		\end{tabular}
		\caption{Flujo de caja para presupuesto alternativo.}
		\label{tabla:sencilla}
	\end{center}
\end{table}

Vamos a calcular el VAN, asumiendo un 8\% del interés cada mes, sobre el flujo anterior:

$P/F_0 = P_0 = -3578,06 $ \\

$P/F_2 = P_1 = \frac{5673,83}{(1+0,08)^1} = 5253,54  $ \\

$P/F_4 = P_2 = \frac{15.025,69}{(1+0,08)^2} = 12882,10  $\\

$VAN = P_0 + P_1 + P_2  = 14557,58$\\

Por lo tanto, con estas condiciones, es altamente rentable realizar el proyecto. Consideramos pues que no es necesario solicitar un préstamo y que podemos asumir el proyecto con la capital de la propia empresa.\\

El TIR se puede calcular con una hoja Excel de la forma que aparece en el siguiente \href{https://support.office.com/es-es/article/tir-funci%C3%B3n-tir-64925eaa-9988-495b-b290-3ad0c163c1bc}{Enlace}. \cite{misc:TIR} \\
	
Tenemos un 21\%. Por lo que obtendríamos un buen porcenatje de beneficio a partir de nuestra inversión por parte de la empresa.


\newpage
\bibliographystyle{plain}
\bibliography{biblio}

\end{document}       
%---------------------------------------------------
