\input{preambuloSimple.tex}
%	TÍTULO Y DATOS DEL ALUMNO
%----------------------------------------------------------------------------------------

\title{	
	\normalfont \normalsize 
	\textsc{\textbf{Planificación y Gestión de Proyectos Informáticos (2018-2019)} \\ Máster Profesional de Ingeniería Informática \\ Universidad de Granada} \\ [25pt] % Your university, school and/or department name(s)
	\horrule{0.5pt} \\[0.4cm] % Thin top horizontal rule
	\huge Propuestas \\ % The assignment title
	\horrule{2pt} \\[0.5cm] % Thick bottom horizontal rule
}

\author{Alejandro Campoy Nieves \\ Luis Gallego Quero} % Nombre y apellidos y correo
\date{\normalsize\today} % Incluye la fecha actual

\usepackage[spanish, es-tabla]{babel}
\usepackage{hyperref} % Para añadir los hiperenlaces.
\hypersetup{
	colorlinks=true,
	linkcolor=blue,
	filecolor=magenta,      
	urlcolor=blue,
}
\usepackage{graphicx}
\usepackage{amssymb, amsmath, amsbsy}
\usepackage{mathptmx}	
\usepackage{float}
\usepackage{booktabs}					%paquete para realización de tablas profesionales
\usepackage{eurosym}

\usepackage[table]{xcolor}
\usepackage{color}
\usepackage{colortbl}
\usepackage{multicol}
\usepackage{multirow}
\usepackage{booktabs}
\usepackage{tabularx}
%---- Paquete para aliniación de texto verticalmente dentro de tablas
\usepackage{array}
%---- Paquetes para los pies de las tablas
\usepackage{caption}
\usepackage{subcaption}


%----------------------------------------------------------------------------------------
% DOCUMENTO
%----------------------------------------------------------------------------------------

\begin{document}
	\maketitle % Muestra el Título
	
	\newpage %inserta un salto de página
	
	%\tableofcontents % para generar el índice de contenidos
	
	%\listoffigures
	
	%\listoftables	
	
	%\newpage
	
	Las propuestas de mejoras que hemos consensuado para esta asignatura son:
	
	\begin{itemize}
		\item  Este problema no es solo de esta asignatura, se podría decir que va dirigida al Máster en general. La cantidad de materia que se da es excesivo en todas las asignaturas. Consideramos que es conveniente replantearse la cantidad de plazos que hay.
		
		\item Al comienzo del curso se dan un plazo de un mes para las primeras tareas de teoría y prácticas. Consideramos que sería conveniente adelantar esta entrega debido a que los estudiantes desconocemos el nivel de estrés al que vamos a ser sometidos durante el curso.
		
		\item Sería aconsejable ir asignando notas a medida que se van realizando las prácticas, ya que no tenemos ninguna retroalimentación de nuestro trabajo y no sabemos hasta que punto estamos realizando las cosas correctamente.
	\end{itemize}

%\newpage
%\bibliographystyle{plain}
%\bibliography{biblio}

\end{document}       
%---------------------------------------------------
