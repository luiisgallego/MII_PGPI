%%%%%%%%%%%%%%%%%%%%%%%%%%%%%%%%%%%%%%%%%
% Short Sectioned Assignment LaTeX Template Version 1.0 (5/5/12)
% This template has been downloaded from: http://www.LaTeXTemplates.com
% Original author:  Frits Wenneker (http://www.howtotex.com)
% License: CC BY-NC-SA 3.0 (http://creativecommons.org/licenses/by-nc-sa/3.0/)
%%%%%%%%%%%%%%%%%%%%%%%%%%%%%%%%%%%%%%%%%

%----------------------------------------------------------------------------------------
%	PACKAGES AND OTHER DOCUMENT CONFIGURATIONS
%----------------------------------------------------------------------------------------

\documentclass[paper=a4, fontsize=11pt]{scrartcl} % A4 paper and 11pt font size

% ---- Entrada y salida de texto -----

\usepackage[T1]{fontenc} % Use 8-bit encoding that has 256 glyphs
\usepackage[utf8]{inputenc}
%\usepackage{fourier} % Use the Adobe Utopia font for the document - comment this line to return to the LaTeX default

% ---- Idioma --------

\usepackage[spanish, es-tabla]{babel} % Selecciona el español para palabras introducidas automáticamente, p.ej. "septiembre" en la fecha y especifica que se use la palabra Tabla en vez de Cuadro

% ---- Otros paquetes ----

\usepackage{url} % ,href} %para incluir URLs e hipervínculos dentro del texto (aunque hay que instalar href)
\usepackage{amsmath,amsfonts,amsthm} % Math packages
%\usepackage{graphics,graphicx, floatrow} %para incluir imágenes y notas en las imágenes
\usepackage{graphics,graphicx, float} %para incluir imágenes y colocarlas

% Para hacer tablas comlejas
%\usepackage{multirow}
%\usepackage{threeparttable}

%\usepackage{sectsty} % Allows customizing section commands
%\allsectionsfont{\centering \normalfont\scshape} % Make all sections centered, the default font and small caps

\usepackage{fancyhdr} % Custom headers and footers
\pagestyle{fancyplain} % Makes all pages in the document conform to the custom headers and footers
\fancyhead{} % No page header - if you want one, create it in the same way as the footers below
\fancyfoot[L]{} % Empty left footer
\fancyfoot[C]{} % Empty center footer
\fancyfoot[R]{\thepage} % Page numbering for right footer
\renewcommand{\headrulewidth}{0pt} % Remove header underlines
\renewcommand{\footrulewidth}{0pt} % Remove footer underlines
\setlength{\headheight}{13.6pt} % Customize the height of the header

\numberwithin{equation}{section} % Number equations within sections (i.e. 1.1, 1.2, 2.1, 2.2 instead of 1, 2, 3, 4)
\numberwithin{figure}{section} % Number figures within sections (i.e. 1.1, 1.2, 2.1, 2.2 instead of 1, 2, 3, 4)
\numberwithin{table}{section} % Number tables within sections (i.e. 1.1, 1.2, 2.1, 2.2 instead of 1, 2, 3, 4)

\setlength\parindent{0pt} % Removes all indentation from paragraphs - comment this line for an assignment with lots of text

\newcommand{\horrule}[1]{\rule{\linewidth}{#1}} % Create horizontal rule command with 1 argument of height

%	TÍTULO Y DATOS DEL ALUMNO
%----------------------------------------------------------------------------------------

\title{	
	\normalfont \normalsize 
	\textsc{\textbf{Planificación y Gestión de Proyectos Informáticos (2018-2019)} \\ Máster Profesional de Ingeniería Informática \\ Universidad de Granada} \\ [25pt] % Your university, school and/or department name(s)
	\horrule{0.5pt} \\[0.4cm] % Thin top horizontal rule
	\huge Retrospectiva \\ % The assignment title
	\horrule{2pt} \\[0.5cm] % Thick bottom horizontal rule
}

\author{Alejandro Campoy Nieves \\ Luis Gallego Quero} % Nombre y apellidos y correo
\date{\normalsize\today} % Incluye la fecha actual

\usepackage[spanish, es-tabla]{babel}
\usepackage{hyperref} % Para añadir los hiperenlaces.
\hypersetup{
	colorlinks=true,
	linkcolor=blue,
	filecolor=magenta,      
	urlcolor=blue,
}
\usepackage{graphicx}
\usepackage{amssymb, amsmath, amsbsy}
\usepackage{mathptmx}	
\usepackage{float}
\usepackage{booktabs}					%paquete para realización de tablas profesionales
\usepackage{eurosym}

\usepackage[table]{xcolor}
\usepackage{color}
\usepackage{colortbl}
\usepackage{multicol}
\usepackage{multirow}
\usepackage{booktabs}
\usepackage{tabularx}
%---- Paquete para aliniación de texto verticalmente dentro de tablas
\usepackage{array}
%---- Paquetes para los pies de las tablas
\usepackage{caption}
\usepackage{subcaption}


%----------------------------------------------------------------------------------------
% DOCUMENTO
%----------------------------------------------------------------------------------------

\begin{document}
	\maketitle % Muestra el Título
	
	\newpage %inserta un salto de página
	
	%\tableofcontents % para generar el índice de contenidos
	
	%\listoffigures
	
	%\listoftables	
	
	%\newpage	
 
\begin{enumerate}
	\item \textbf{¿Cuál era el plan al comienzo del proyecto? ¿Cómo cambió a lo largo de su ejecución?} \\
	
		   Mejorar la calidad de los diagnósticos de las personas, utilizando Deep Learning para el reconocimiento de fracturas. Principalmente cambió el aspecto de planificación y gestión del proyecto en todos los sentidos.
		   
	\item \textbf{¿Qué sabe ahora que le hubiese gustado saber al comienzo del proyecto? ¿Cómo habría cambiado el proyecto de haberlo sabido antes?} \\
	
		  Los requisitos del cliente es algo que interesa saber de forma clara desde el principio, cosa que prácticamente nunca ocurre. Hubieramos enfocado el proyecto de una forma más directa a estos requisitos y hubiéramos perdido menos recursos en ese proceso.
		  
	\item \textbf{¿Qué aspectos fueron especialmente bien durante el proyecto? ¿Por qué?} \\
	
				El trabajo en equipo y la coordinación entre los integrantes del proyecto. Nos conocemos desde hace tiempo, no solo en el ámbito personal sino en el profesional también.
				
	\item \textbf{¿Qué aspectos salieron rematadamente mal durante el proyecto? ¿Por qué?} \\
	
				Principalmente la estimación de presupuestos. Ya que es algo sobre lo que no tenemos nada de experiencia.
				
	\item \textbf{¿Qué fases o etapas del proyecto habrían necesitado más tiempo para poder ejecutarse de una forma más adecuada, teniendo en cuenta las restricciones temporales ya impuestas?} \\
	
				Principalmente, la planificación del proyecto y la reuniones con el cliente. Ya que es una parte muy importante que está directamente relacionada con la calidad del producto final.
				
	\item \textbf{¿En qué fases se tuvieron que repetir tareas ya realizadas (i.e. “rework” considerado innecesario)? ¿Cómo podría haberse evitado?}\\
	
				Una vez más, las reuniones con el cliente. No para obtener nuevos requisitos, sino para aclarar los que ya habían sido captados. El principal problema reside en que el cliente puede cambiar de opinión y esto puede ralentizar el proceso y desandar lo avanzado.
				
	\item \textbf{¿Qué herramientas se utilizaron durante el proyecto? ¿En qué funcionaron bien? ¿En qué funcionaron mal? ¿Cómo cambiaría el uso de herramientas de cara a proyectos futuros?} \\
	
	Todas las mencionadas en las prácticas anteriores de esta asignatura. Los algoritmos de Deep Learning son de gran ayuda para cumplir nuestro objetivo y un pilar fundamental para el éxito. Cuanto más se entiende este tipo de herramientas, son más sencillas de mejorar para nuestro problema concreto.
	
	\item \textbf{¿Fue efectivo el proceso utilizado a lo largo del proyecto? ¿En qué aspectos funcionó razonablemente bien? ¿En qué aspectos habría que mejorarlo?} \\
	
	El proyecto se ha realizado utilizando la metodología Scrum, con la que ya teníamos conocimiento de cómo funcionaba. Esta técnica aporta muchas ventajas tales como alta adaptabilidad a los posibles cambios en los requisitos del cliente o nuevos requisitos. 
	
	\item \textbf{¿Cuáles son los aspectos más relevantes de este proyecto que resaltaría para compartirlos con el gestor del proyecto y el resto de su equipo de cara a proyectos futuros?} \\ 
	
	Uno de los aspectos que más relevancia pueden tener en cualquier proyecto es la planificación. En definitiva, el tiempo que se emplea en este aspecto no se debe considerar como ``tiempo perdido'', ya que será decisivo y puede hacernos ahorrar muchos recursos y esfuerzo o generar justo lo contrario.
		
\end{enumerate}

Las propuestas de \textbf{mejoras para el futuro} son:

\begin{enumerate}
	\item Un mayor número de empleados para poder hacer una planificación y asignación de tareas más dinámica y paralela.
	\item Mejora en precisión de diagnósticos a los pacientes.
	\item Que nuestro sistema sea capaz de clasificar un grupo más amplio de problemas tales como esguinces, microrroturas.
\end{enumerate}

%\newpage
%\bibliographystyle{plain}
%\bibliography{biblio}

\end{document}       
%---------------------------------------------------
