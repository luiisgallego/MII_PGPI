\input{preambuloSimple.tex}
%	TÍTULO Y DATOS DEL ALUMNO
%----------------------------------------------------------------------------------------

\title{	
	\normalfont \normalsize 
	\textsc{\textbf{Planificación y Gestión de Proyectos Informáticos (2018-2019)} \\ Máster Profesional de Ingeniería Informática \\ Universidad de Granada} \\ [25pt] % Your university, school and/or department name(s)
	\horrule{0.5pt} \\[0.4cm] % Thin top horizontal rule
	\huge Criterios de Selección de Personal \\ % The assignment title
	\horrule{2pt} \\[0.5cm] % Thick bottom horizontal rule
}

\author{Alejandro Campoy Nieves \\ Luis Gallego Quero} % Nombre y apellidos y correo
\date{\normalsize\today} % Incluye la fecha actual

\usepackage[spanish, es-tabla]{babel}
\usepackage{hyperref} % Para añadir los hiperenlaces.
\hypersetup{
	colorlinks=true,
	linkcolor=blue,
	filecolor=magenta,      
	urlcolor=blue,
}
\usepackage{graphicx}
\usepackage{amssymb, amsmath, amsbsy}
\usepackage{mathptmx}	
\usepackage{float}
\usepackage{booktabs}					%paquete para realización de tablas profesionales
\usepackage{eurosym}

\usepackage[table]{xcolor}
\usepackage{color}
\usepackage{colortbl}
\usepackage{multicol}
\usepackage{multirow}
\usepackage{booktabs}
\usepackage{tabularx}
%---- Paquete para aliniación de texto verticalmente dentro de tablas
\usepackage{array}
%---- Paquetes para los pies de las tablas
\usepackage{caption}
\usepackage{subcaption}


%----------------------------------------------------------------------------------------
% DOCUMENTO
%----------------------------------------------------------------------------------------

\begin{document}
	\maketitle % Muestra el Título
	
	\newpage %inserta un salto de página
	
	\tableofcontents % para generar el índice de contenidos
	
	%\listoffigures
	
	\listoftables	
	
	\newpage	
 
\section{Oferta de Trabajo}

Si te interesa trabajar en un equipo de desarrollo con una gran capacidad de evolución y aportación a la ciencia, esta es tu oportunidad. Como pequeña empresa, somos un startup formada provisionalmente por dos integrantes con la motivación de desarrollar un proyecto sobre diagnósticos automáticos de fracturas a partir de resonancias magnéticas. \\

Buscamos a un ingeniero informático, con buena capacidad de programación y conocimiento en Deep Learning. Los requisitos fundamentales que se buscan son:

\begin{itemize}
	\item Estudios mínimos: grado en ingeniería informática.
	\item Experiencia mínima: No se requiere experiencia laboral previa.	
	\item Conocimientos necesarios:
		\begin{itemize}
			\item Python.
			\item Programación orientada a objetos.
			\item Conocimiento en el uso de APIs.
			\item Desarrollo de interfaces.
			\item Git y GitHub.
			\item Metodologías Ágiles de desarrollo.
			\item Tecnologías de la información.
		\end{itemize}
	\item Requisitos mínimos:
	\begin{itemize}
		\item Experiencia en Deep Learning muy recomendable.
		\item Experiencia en análisis de requisitos funcionales y no funcionales.
		\item Experiencia con herramientas de trabajo colaborativo.
		\item Disponibilidad inmediata.
		\item Inglés nivel B1.
	\end{itemize}
\end{itemize}

Se oferta:

\begin{itemize}
	\item Puestos de trabajo: 1.
	\item Salario: 1.700 \euro mensuales, 20.400 \euro anuales.
	\item Jornada laboral: 35 horas a la semana, desde 8:00 hasta 15:00 de Lunes a Viernes.
\end{itemize}

\newpage

\section{Preselección de Candidatos}

Para realizar una buena preselección de candidatos, se han clasificado los requisitos con valores dentro del intervalo 0 y 5 ambos incluidos. De tal manera que definimos el grado de relevancia que supone para nosotros cumplir estos factores. Posteriormente, todo candidato que supere el 3 sobre 5 de media será llamado para una entrevista personal.

\begin{table}[H]
	\begin{center}
		\begin{tabular}{|l||c|}
			\hline 
			Criterio & Valoración \\
			\hline \hline
			Python & 3 \\ \hline
			Programación orientada a objetos & 2 \\ \hline
			Conocimiento en el uso de APIs & 2 \\ \hline
			Desarrollo de interfaces & 1 \\ \hline
			Git y GitHub & 2 \\ \hline
			Metodologías Ágiles de desarrollo & 3 \\ \hline
			Tecnologías de la información & 2 \\ \hline
			Experiencia en Deep Learning & 5 \\ \hline
			Experiencia en análisis de requisitos funcionales y no funcionales & 3 \\ \hline
			Experiencia con herramientas de trabajo colaborativo & 2 \\ \hline
			Disponibilidad inmediata & 4 \\ \hline
			Inglés nivel B1 & 4 \\ \hline
		\end{tabular}
		\caption{Relevancia de requisitos para preselección de candidatos.}
		\label{tabla:preseleccion}
	\end{center}
\end{table}

\section{Entrevistas con los candidatos}

En primer lugar se realizará una entrevista previa al candidato. De esta forma se obtiene la posibilidad, no solo de saber su experiencia básica, sino de observar su forma de ser y la forma de reaccionar que tiene. \\

La idea fundamental es el desarrollo de una ficha de tal forma que pueda ser rellenada por la persona que entreviste a esa persona, para poder recopilar esta información de una forma estandarizada y organizada.


\begin{table}[H]
	\begin{center}
		\begin{tabular}{|p{4cm}|p{10cm}|}
			\hline 
			\multicolumn{2}{|c|}{\textbf{Ficha entrevista}} \\
			\hline
			Experiencia laboral & \parbox[l][0.3\textwidth][c]{8cm}{
				\begin{itemize}
					\item[$\square$] Sin experiencia.
					\item[$\square$] Inferior a 2 años.
					\item[$\square$] Entre 2 y 4 años.
					\item[$\square$] Superior a 4 años.
			\end{itemize} } \\ \hline
			Estudios realizados & \parbox[l][0.4\textwidth][c]{8cm}{
				\begin{itemize}
					\item[$\square$] Ingeniería informática.
					\item[$\square$] Otra carrera relacionada.
					\item[$\square$] Grado superior.
					\item[$\square$] Grado medio.
					\item[$\square$] Otros estudios.
			\end{itemize} } \\ \hline
			Referencia personal & \parbox[l][0.3\textwidth][c]{8cm}{
				\begin{itemize}
					\item[$\square$] Trabajos individuales.
					\item[$\square$] Trabajos en equipo.
					\item[$\square$] Aportaciones software libre.
					\item[$\square$] Etapa en el extranjero.
			\end{itemize} } \\ \hline
		Capacidades & \parbox[l][0.4\textwidth][c]{8cm}{
			\begin{itemize}
				\item[$\square$] Desarrollo de algoritmos de Deep Learning.
				\item[$\square$] Conocimiento en el uso de APIs.
				\item[$\square$] Desarrollo de interfaces.
				\item[$\square$] Tecnologías de la información.
				\item[$\square$] Idiomas.
				\item[$\square$] Otros.
		\end{itemize} } \\ \hline
		\end{tabular}
	\end{center}
\end{table}
\newpage
\begin{table}[H]
	\begin{center}
		\begin{tabular}{|p{4cm}|p{10cm}|}
			\hline 
				Motivación & \parbox[l][0.6\textwidth][c]{8cm}{
				\begin{itemize}
					\item ¿Por qué busca este trabajo?
					\begin{itemize}
						\item[$\square$] Necesidad de trabajo.
						\item[$\square$] Necesidad de experiencia.
						\item[$\square$] Formación en el área.
						\item[$\square$] Interés personal.
						\item[$\square$] Aporte a la sanidad.
						\item[$\square$] Otras.
					\end{itemize}
					\item ¿Perspectivas de futuro?
					\begin{itemize}
						\item[$\square$] Mejorar su vida laboral.
						\item[$\square$] Nuevos proyectos.
						\item[$\square$] Mejora de rango en la empresa.
						\item[$\square$] Otros.
					\end{itemize}
				\end{itemize} } \\ \hline
				Cuestiones varias & \parbox[l][0.2\textwidth][c]{8cm}{
					\begin{itemize}
						\item[$\square$] Salario.
						\item[$\square$] Disponibilidad.
						\item[$\square$] Amoldación a jornada.
				\end{itemize} } \\ \hline
				Cuestiones & \parbox[l][0.2\textwidth][c]{8cm}{
									 } \\ \hline
		\end{tabular}
		\caption{Ficha para entrevistas a candidatos}
		\label{tabla:ficha}
	\end{center}
\end{table}

\section{Ranking final de candidatos}

Después de haber pasado ciertos filtros, nos centramos en el punto de vista más técnico, de una forma más enfocada y especifica hacia nuestro proyecto.

\newpage

\begin{table}[H]
	\begin{center}
		\begin{tabular}{|p{3.5cm}|p{8cm}|p{3.5cm}|}
			\hline 
			\multicolumn{3}{|c|}{\textbf{Evaluación}} \\
			\hline
			\multicolumn{1}{|c|}{Tema} & \multicolumn{1}{c|}{Opciones} & \multicolumn{1}{c|}{Nota} \\
			\hline
			Experiencia laboral & \parbox[l][0.3\textwidth][c]{8cm}{
				\begin{itemize}
					\item Sin experiencia (0).
					\item Inferior a 2 años (2).
					\item Entre 2 y 4 años (3).
					\item Superior a 4 años (5).
			\end{itemize} } & \\ \hline
			Estudios realizados & \parbox[l][0.4\textwidth][c]{8cm}{
				\begin{itemize}
					\item Ingeniería informática (5).
					\item Otra carrera relacionada (3).
					\item Grado superior (1).
					\item Grado medio (0).
					\item Otros estudios.
			\end{itemize} } & \\ \hline
			Referencia personal & \parbox[l][0.3\textwidth][c]{8cm}{
				\begin{itemize}
					\item Trabajos individuales (2).
					\item Trabajos en equipo (2).
					\item Aportaciones software libre (4).
					\item Etapa en el extranjero (3).
			\end{itemize} } & \\ \hline
			Capacidades & \parbox[l][0.4\textwidth][c]{8cm}{
				\begin{itemize}
					\item Desarrollo de algoritmos de Deep Learning (5).
					\item Conocimiento en el uso de APIs (4).
					\item Desarrollo de interfaces (3).
					\item Tecnologías de la información (2).
					\item Idiomas (2).
					\item Otros.
			\end{itemize} } & \\ \hline
		\end{tabular}
	\end{center}
\end{table}
\newpage
\begin{table}[H]
	\begin{center}
		\begin{tabular}{|p{3.5cm}|p{8cm}|p{3.5cm}|}
			\hline 
			Motivación & \parbox[l][0.6\textwidth][c]{8cm}{
				\begin{itemize}
					\item ¿Por qué busca este trabajo?
					\begin{itemize}
						\item Necesidad de trabajo (1).
						\item Necesidad de experiencia (1).
						\item Formación en el área (1).
						\item Interés personal (3).
						\item Aporte a la sanidad (1).
						\item Otras.
					\end{itemize}
					\item ¿Perspectivas de futuro?
					\begin{itemize}
						\item Mejorar su vida laboral (1).
						\item Nuevos proyectos (2).
						\item Mejora de rango en la empresa (1).
						\item Otros.
					\end{itemize}
			\end{itemize} } & \\ \hline
			Cuestiones varias & \parbox[l][0.2\textwidth][c]{8cm}{
				\begin{itemize}
					\item Salario (3).
					\item Disponibilidad (5).
					\item Amoldación a jornada (4).
			\end{itemize} } & \\ \hline
			Cuestiones & \parbox[l][0.2\textwidth][c]{8cm}{
			} & \\ \hline
		\end{tabular}
		\caption{Evaluación de fichas sobre entrevistas a candidatos.}
		\label{tabla:evaluación}
	\end{center}
\end{table}


%\newpage
%\bibliographystyle{plain}
%\bibliography{biblio}

\end{document}       
%---------------------------------------------------
