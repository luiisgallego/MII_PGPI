\input{preambuloSimple.tex}
%	TÍTULO Y DATOS DEL ALUMNO
%----------------------------------------------------------------------------------------

\title{	
	\normalfont \normalsize 
	\textsc{\textbf{Planificación y Gestión de Proyectos Informáticos (2018-2019)} \\ Máster Profesional de Ingeniería Informática \\ Universidad de Granada} \\ [25pt] % Your university, school and/or department name(s)
	\horrule{0.5pt} \\[0.4cm] % Thin top horizontal rule
	\huge Criterios de Selección de Personal \\ % The assignment title
	\horrule{2pt} \\[0.5cm] % Thick bottom horizontal rule
}

\author{Alejandro Campoy Nieves \\ Luis Gallego Quero} % Nombre y apellidos y correo
\date{\normalsize\today} % Incluye la fecha actual

\usepackage[spanish, es-tabla]{babel}
\usepackage{hyperref} % Para añadir los hiperenlaces.
\hypersetup{
	colorlinks=true,
	linkcolor=blue,
	filecolor=magenta,      
	urlcolor=blue,
}
\usepackage{graphicx}
\usepackage{amssymb, amsmath, amsbsy}
\usepackage{mathptmx}	
\usepackage{float}
\usepackage{booktabs}					%paquete para realización de tablas profesionales
\usepackage{eurosym}
%\usepackage[table]{xcolor}
%\definecolor{lightgray}{gray}{0.9}
\usepackage{xcolor}
\usepackage{colortbl}


%----------------------------------------------------------------------------------------
% DOCUMENTO
%----------------------------------------------------------------------------------------

\begin{document}
	\maketitle % Muestra el Título
	
	\newpage %inserta un salto de página
	
	\tableofcontents % para generar el índice de contenidos
	
	%\listoffigures
	
	\listoftables	
	
	\newpage	
 
\section{Oferta de Trabajo}

Si te interesa trabajar en un equipo de desarrollo con una gran capacidad de evolución y aportación a la ciencia, esta es tu oportunidad. Como pequeña empresa, somos un startup formada provisionalmente por dos integrantes con la motivación de desarrollar un proyecto sobre diagnósticos automáticos de fracturas a partir de resonancias magnéticas. \\

Buscamos a un ingeniero informático, con buena capacidad de programación y conocimiento en Deep Learning. Los requisitos fundamentales que se buscan son:

\begin{itemize}
	\item Estudios mínimos: grado en ingeniería informática.
	\item Experiencia mínima: No se requiere experiencia laboral previa.	
	\item Conocimientos necesarios:
		\begin{itemize}
			\item Python.
			\item Programación orientada a objetos.
			\item Conocimiento en el uso de APIs.
			\item Desarrollo de interfaces.
			\item Git y GitHub.
			\item Metodologías Ágiles de desarrollo.
			\item Tecnologías de la información.
		\end{itemize}
	\item Requisitos mínimos:
	\begin{itemize}
		\item Experiencia en Deep Learning muy recomendable.
		\item Experiencia en análisis de requisitos funcionales y no funcionales.
		\item Experiencia con herramientas de trabajo colaborativo.
		\item Disponibilidad inmediata.
		\item Inglés nivel B1.
	\end{itemize}
\end{itemize}

Se oferta:

\begin{itemize}
	\item Puestos de trabajo: 1.
	\item Salario: 1.700 \euro mensuales, 20.400 \euro anuales.
	\item Jornada laboral: 35 horas a la semana, desde 8:00 hasta 15:00 de Lunes a Viernes.
\end{itemize}

\newpage

\section{Preselección de Candidatos}


\newpage
\bibliographystyle{plain}
\bibliography{biblio}

\end{document}       
%---------------------------------------------------
