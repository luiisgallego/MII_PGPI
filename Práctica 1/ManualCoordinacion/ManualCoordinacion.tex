\documentclass[a4paper,12pt,oneside]{article}

\usepackage[spanish, es-tabla]{babel}
\usepackage[hidelinks]{hyperref}
\usepackage{graphicx}
\usepackage{amssymb, amsmath, amsbsy}
\usepackage{mathptmx}	
\usepackage{float}

\title{Manual de Coordinación}
\author{Luis Gallego Quero \\ Alejandro Campoy Nieves}

\date{\today}

\begin{document}
\maketitle			
 
\section{Ciclo de vida}

El ciclo de vida que vamos a utilizar es iterativo. Se trata de ir obteniendo parte del producto por pequeños bloques a los que se les denominan ciclos de desarrollo o iteraciones. La razón de esto es ir obteniendo desde el primer momento un proyecto funcional. Aunque en las primeras etapas el programa no cumple las expectativas esperadas y no sea fiable dando resultados de calidad dudosa, tenemos la posibilidad de obtener una aplicación útil en la que podemos ir viendo los resultados iniciales, valorarlos y pensar o analizar la estrategia para la siguiente iteración.

\section{Metodología de desarrollo}

Ligado al punto anterior, utilizaremos una metodología ágil basada en SCRUM. Cuyo objetivo sea gestionar y planificar el proyecto con posibilidades de cambio a última hora. 

\section{Recursos software desarrollo}

Desarrollaremos este proyecto a través de Spyder, un IDE para Python que nos ayudará al desarollo de este sistema. Incluyendo las librerías y funcionalidades disponibles que nos permitan tener una base en la que montar este proyecto y que nos aporte una mejora en la calidad y facilidad de trabajo. \\

Para la realización de los distintos diagramas utilizaremos las herramientas que nos proporciona Draw.io, Microsoft Visio y github como gestor de documentación y controlador de versiones.

\section{Organización del equipo de trabajo}

La organización que hemos consensuado es de tipo dinámico. Esto quiere decir, que los integrantes de este proyecto desarrollan roles temporales durante un intervalo variable. En resumen, una persona se dedica a supervisar el proyecto principalmente, mientras que la otra lo desarrolla. Estos papeles se intercambian con el objetivo de no saturar las capacidades de trabajo de cada uno.

\section{Herramientas para comunicaciones en el equipo de trabajo}

\begin{itemize}
	\item Github como controlador de versiones.
	\item Trello para el desarrollo de SCRUM.
	\item Whatsapp para la comunicación entre los integrantes del proyecto.
	\item Hangouts para posibles videoconferencias.
\end{itemize}

\section{Relaciones con el cliente}

\begin{enumerate}
	\item Entrevista: Primer encuentro con el cliente destinado en esencia al análisis de requisitos del proyecto que abordamos. 
	\item Reuniones: Siendo la base de SCRUM realizar reuniones periódicas con el cliente, de este modo, podemos presentarle los diferentes avances y, en ese momento, se puede perfilarse distintos aspectos del proyecto con el objetivo de ajustarse a las necesidades del cliente en cuestión.
\end{enumerate}

\section{Estándares de documentación}

\begin{itemize}
	\item Como compositor de texto, utilizaremos la herramienta llamada \LaTeX.
	\item De portada utilizaremos siempre la misma plantilla para darle homogeneidad al proyecto.
	\item Se enumeraran las páginas de forma automática sin contar la portada, de tal forma que se desarrolla un índice acorde a cada parte de la documentación.
	\item Aunque el texto se escriba en formato \LaTeX, la documentación final del proyecto será presentada en el formato PDF.
\end{itemize}

\section{estándares de código}

\begin{itemize}
	\item Como vamos a realizar el proyecto en Python, utilizaremos los convenios de presentación típicos de este lenguaje (en las variables, primera letra minuscula y resto de palabras primera letra en mayúscula, por ejemplo).
	\item Los comentarios serán especificados en castellano siguiendo el formato de comentarios de Python para los distinto módulos.
\end{itemize}


\section{Control de versiones}

Tanto para los documentos como el código generado durante el desarrollo de este proyecto, serán almacenados, controlados y dirigidos a través de Github.

\section{Gestión de calidad}

Los test para realizar las correspondientes pruebas de desarrollo serán elaboradas a través de la herramienta de Python llamada Unittest.\\

También se utilizará como punto de apoyo, las verificaciones y visto bueno por parte del cliente, que será considerado como un punto más del proceso de prueba y testeo.



%%%%%% RECURSOS %%%%%%




% \textbf{negrita}
%  \textit{cursiva}

% Insertar imagen
%\begin{figure}[h]
	%\centering
		%\includegraphics[scale=0.1]{./ruta}
	%\caption{Texto parte inferior}
	%\label{etiqueta para referencia}
%\end{figure}

% 		LISTADO DE ITEMS
%\begin{itemize}
	%\item 
%\end{itemize}

% 		BIBLIOGRAFÍA
%\begin{thebibliography}{X}
	%\bibitem{nombreItem} \url{enlace} 
%\end{thebibliography}


\end{document}       
%---------------------------------------------------
