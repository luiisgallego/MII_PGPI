\input{preambuloSimple.tex}
%	TÍTULO Y DATOS DEL ALUMNO
%----------------------------------------------------------------------------------------

\title{	
	\normalfont \normalsize 
	\textsc{\textbf{Planificación y Gestión de Proyectos Informáticos (2018-2019)} \\ Máster Profesional de Ingeniería Informática \\ Universidad de Granada} \\ [25pt] % Your university, school and/or department name(s)
	\horrule{0.5pt} \\[0.4cm] % Thin top horizontal rule
	\huge Selección de herramientas \\ % The assignment title
	\horrule{2pt} \\[0.5cm] % Thick bottom horizontal rule
}

\author{Alejandro Campoy Nieves \\ Luis Gallego Quero} % Nombre y apellidos y correo
\date{\normalsize\today} % Incluye la fecha actual

\usepackage[spanish, es-tabla]{babel}
\usepackage{hyperref} % Para añadir los hiperenlaces.
\hypersetup{
	colorlinks=true,
	linkcolor=blue,
	filecolor=magenta,      
	urlcolor=blue,
}
\usepackage{graphicx}
\usepackage{amssymb, amsmath, amsbsy}
\usepackage{mathptmx}	
\usepackage{float}
\usepackage{booktabs}					%paquete para realización de tablas profesionales
\usepackage{eurosym}

\usepackage[table]{xcolor}
\usepackage{color}
\usepackage{colortbl}
\usepackage{multicol}
\usepackage{multirow}
\usepackage{booktabs}
\usepackage{tabularx}
%---- Paquete para aliniación de texto verticalmente dentro de tablas
\usepackage{array}
%---- Paquetes para los pies de las tablas
\usepackage{caption}
\usepackage{subcaption}


%----------------------------------------------------------------------------------------
% DOCUMENTO
%----------------------------------------------------------------------------------------

\begin{document}
	\maketitle % Muestra el Título
	
	\newpage %inserta un salto de página
	
	\tableofcontents % para generar el índice de contenidos
	
	%\listoffigures
	
	%\listoftables	
	
	\newpage	
 
\section{Control de versiones}

El control de versiones por excelencia es Git, y para nuestro proyecto no hay duda de que usaremos ese. Además, hay una buena cantidad de plataformas que hacen uso de él, como Github, las cuales son fáciles e intuitivas de usar. Una apuesta segura.

\section{Compilación}

Eclipse. Aunque las comparativas han quedado bastante igualadas, para nuestro proyecto el mejor IDE y compilador es Eclipse. Además, su fácil integración con herramientas como Apache por ejemplo, son ideales para el uso de redes neuronales. 

\section{Automatización de pruebas}

En este caso nos quedamos con JMeter básicamente  por una razón, es la única de las tres herramientas que te permite utilizarla con diferentes aplicaciones, servidores o protocolos. Las restricciones que establecen las demás herramientas no son  útiles para proyectos que hacen uso de lenguajes y herramientas variadas.

\section{Integración continua}

Bamboo. Principalmente por dos razones, nos ofrece un precio muy competitivo respecto a las demás herramientas y además tiene un alto nivel de funcionalidad. Destacar también que está disponible para cualquier lenguaje.

\section{Seguimiento de errores/defectos}

Para esta última elección las comparativas también han sido determinantes, siendo la herramienta elegida GitLab. Su coste gratuito y su fácil uso son características más que deseables. Además, podemos encontrar una amplia comunidad en torno a ella.

%\newpage
%\bibliographystyle{plain}
%\bibliography{biblio}

\end{document}       
%---------------------------------------------------
