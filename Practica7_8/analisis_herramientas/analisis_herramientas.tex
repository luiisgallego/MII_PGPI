%%%%%%%%%%%%%%%%%%%%%%%%%%%%%%%%%%%%%%%%%
% Short Sectioned Assignment LaTeX Template Version 1.0 (5/5/12)
% This template has been downloaded from: http://www.LaTeXTemplates.com
% Original author:  Frits Wenneker (http://www.howtotex.com)
% License: CC BY-NC-SA 3.0 (http://creativecommons.org/licenses/by-nc-sa/3.0/)
%%%%%%%%%%%%%%%%%%%%%%%%%%%%%%%%%%%%%%%%%

%----------------------------------------------------------------------------------------
%	PACKAGES AND OTHER DOCUMENT CONFIGURATIONS
%----------------------------------------------------------------------------------------

\documentclass[paper=a4, fontsize=11pt]{scrartcl} % A4 paper and 11pt font size

% ---- Entrada y salida de texto -----

\usepackage[T1]{fontenc} % Use 8-bit encoding that has 256 glyphs
\usepackage[utf8]{inputenc}
%\usepackage{fourier} % Use the Adobe Utopia font for the document - comment this line to return to the LaTeX default

% ---- Idioma --------

\usepackage[spanish, es-tabla]{babel} % Selecciona el español para palabras introducidas automáticamente, p.ej. "septiembre" en la fecha y especifica que se use la palabra Tabla en vez de Cuadro

% ---- Otros paquetes ----

\usepackage{url} % ,href} %para incluir URLs e hipervínculos dentro del texto (aunque hay que instalar href)
\usepackage{amsmath,amsfonts,amsthm} % Math packages
%\usepackage{graphics,graphicx, floatrow} %para incluir imágenes y notas en las imágenes
\usepackage{graphics,graphicx, float} %para incluir imágenes y colocarlas

% Para hacer tablas comlejas
%\usepackage{multirow}
%\usepackage{threeparttable}

%\usepackage{sectsty} % Allows customizing section commands
%\allsectionsfont{\centering \normalfont\scshape} % Make all sections centered, the default font and small caps

\usepackage{fancyhdr} % Custom headers and footers
\pagestyle{fancyplain} % Makes all pages in the document conform to the custom headers and footers
\fancyhead{} % No page header - if you want one, create it in the same way as the footers below
\fancyfoot[L]{} % Empty left footer
\fancyfoot[C]{} % Empty center footer
\fancyfoot[R]{\thepage} % Page numbering for right footer
\renewcommand{\headrulewidth}{0pt} % Remove header underlines
\renewcommand{\footrulewidth}{0pt} % Remove footer underlines
\setlength{\headheight}{13.6pt} % Customize the height of the header

\numberwithin{equation}{section} % Number equations within sections (i.e. 1.1, 1.2, 2.1, 2.2 instead of 1, 2, 3, 4)
\numberwithin{figure}{section} % Number figures within sections (i.e. 1.1, 1.2, 2.1, 2.2 instead of 1, 2, 3, 4)
\numberwithin{table}{section} % Number tables within sections (i.e. 1.1, 1.2, 2.1, 2.2 instead of 1, 2, 3, 4)

\setlength\parindent{0pt} % Removes all indentation from paragraphs - comment this line for an assignment with lots of text

\newcommand{\horrule}[1]{\rule{\linewidth}{#1}} % Create horizontal rule command with 1 argument of height

%	TÍTULO Y DATOS DEL ALUMNO
%----------------------------------------------------------------------------------------

\title{	
	\normalfont \normalsize 
	\textsc{\textbf{Planificación y Gestión de Proyectos Informáticos (2018-2019)} \\ Máster Profesional de Ingeniería Informática \\ Universidad de Granada} \\ [25pt] % Your university, school and/or department name(s)
	\horrule{0.5pt} \\[0.4cm] % Thin top horizontal rule
	\huge Análisis de herramientas \\ % The assignment title
	\horrule{2pt} \\[0.5cm] % Thick bottom horizontal rule
}

\author{Alejandro Campoy Nieves \\ Luis Gallego Quero} % Nombre y apellidos y correo
\date{\normalsize\today} % Incluye la fecha actual

\usepackage[spanish, es-tabla]{babel}
\usepackage{hyperref} % Para añadir los hiperenlaces.
\hypersetup{
	colorlinks=true,
	linkcolor=blue,
	filecolor=magenta,      
	urlcolor=blue,
}
\usepackage{graphicx}
\usepackage{amssymb, amsmath, amsbsy}
\usepackage{mathptmx}	
\usepackage{float}
\usepackage{booktabs}					%paquete para realización de tablas profesionales
\usepackage{eurosym}

\usepackage[table]{xcolor}
\usepackage{color}
\usepackage{colortbl}
\usepackage{multicol}
\usepackage{multirow}
\usepackage{booktabs}
\usepackage{tabularx}
%---- Paquete para aliniación de texto verticalmente dentro de tablas
\usepackage{array}
%---- Paquetes para los pies de las tablas
\usepackage{caption}
\usepackage{subcaption}


%----------------------------------------------------------------------------------------
% DOCUMENTO
%----------------------------------------------------------------------------------------

\begin{document}
	\maketitle % Muestra el Título
	
	\newpage %inserta un salto de página
	
	\tableofcontents % para generar el índice de contenidos
	
	%\listoffigures
	
	\listoftables	
	
	\newpage	
 
\section{Introducción}

En este documento, se realizará un análisis de las herramientas existentes para facilitar distintos aspectos de la gestión de cambios de un proyecto de desarrollo de software. 

\section{Control de versiones}

\subsection{Git}

\begin{itemize}
	\item \textbf{Nombre de la herramienta}: Git
	\item \textbf{URL}: \url{https://git-scm.com/}
	\item \textbf{Coste}: Gratuito.
	\item \textbf{Características destacadas}:
		\begin{enumerate}
			\item Escrito en una combinación de Perl, C y varios scripts de shell.
			\item Descentralización, rápido, flexible y robusto.
			\item Sencillo de aprender a utilizar.
		\end{enumerate}
	\item \textbf{Limitaciones identificadas}:
		\begin{enumerate}
			\item Gran número de archivos.
			\item Archivos de gran tamaño.
			\item Paquetes enormes.
		\end{enumerate}
\end{itemize}

\subsection{Bazaar}

\begin{itemize}
	\item \textbf{Nombre de la herramienta}: Bazaar
	\item \textbf{URL}: \url{http://bazaar.canonical.com/en/}
	\item \textbf{Coste}: Gratuito.
	\item \textbf{Características destacadas}:
	\begin{enumerate}
		\item Software Libre y Opensource.
		\item Desarrollado en Python.
		\item Versiones empaquetadas para la mayoría de distribuciones GNU/Linux, Mac OS X y MS Windows.
	\end{enumerate}
	\item \textbf{Limitaciones identificadas}: Tiene algunos bugs como
	\begin{enumerate}
		\item No se puede mezclar una rama vacía.
		\item Bazaar debería tener una caché para versiones descargables.
		\item Problema con codificación ASCII.
	\end{enumerate}
\end{itemize}

\subsection{Mercurial}

\begin{itemize}
	\item \textbf{Nombre de la herramienta}: Mercurial
	\item \textbf{URL}: \url{https://www.mercurial-scm.org/}
	\item \textbf{Coste}: Gratuito
	\item \textbf{Características destacadas}:
	\begin{enumerate}
		\item Curva de aprendizaje más rápida que git.
		\item Gestión distribuida.
		\item Al hacer un cambio, se almacenan dichos cambios (no el fichero al completo).
	\end{enumerate}
	\item \textbf{Limitaciones identificadas}:
	\begin{enumerate}
		\item Gran dependencia de Atlassian.
		\item Comunidad menos amplia que git.
		\item Al añadir extensiones, combinar características y funcionalidad es más difícil. 
	\end{enumerate}
\end{itemize}

\section{Compilación}

\subsection{Eclipse}

\begin{itemize}
	\item \textbf{Nombre de la herramienta}: Eclipse
	\item \textbf{URL}: \url{https://www.eclipse.org/}
	\item \textbf{Coste}: Gratuito.
	\item \textbf{Características destacadas}:
	\begin{enumerate}
		\item Permite gestionar proyectos.
		\item Incluye depurador de código.
		\item Extensa colección de plugins.
	\end{enumerate}
	\item \textbf{Limitaciones identificadas}:
	\begin{enumerate}
		\item Algunos plugins son difíciles de configurar, incluso se encuentran anticuados.
		\item Demasiadas versiones diferentes que pueden resultar confusas.
		\item Cantidad de recursos consumidos excesivos.
	\end{enumerate}
\end{itemize}

\subsection{NetBeans}

\begin{itemize}
	\item \textbf{Nombre de la herramienta}: NetBeans
	\item \textbf{URL}: \url{https://netbeans.org/} 
	\item \textbf{Coste}: Gratuito
	\item \textbf{Características destacadas}:
	\begin{enumerate}
		\item Lenguaje multiplataforma.
		\item Permite desarrollar aplicaciones web dinámicas.
		\item Manejo automático de la memoria.
	\end{enumerate}
	\item \textbf{Limitaciones identificadas}:
	\begin{enumerate}
		\item Lentitud a la hora de ejecutar aplicaciones.
		\item Requiere un intérprete.
		\item Algunas herramientas tienen un costo adicional.
	\end{enumerate}
\end{itemize}

\subsection{Android Studio}

\begin{itemize}
	\item \textbf{Nombre de la herramienta}: Android Studio
	\item \textbf{URL}: \url{https://developer.android.com/studio/}
	\item \textbf{Coste}: Gratuito.
	\item \textbf{Características destacadas}: 
	\begin{enumerate}
		\item Se puede programar para diferentes versiones de Android.
		\item Compilación rápida.
		\item Ejecución del app en tiempo real gracias al emulador.
	\end{enumerate}
	\item \textbf{Limitaciones identificadas}:
	\begin{enumerate}
		\item Entorno algo inestable.
		\item Los requisitos son un poco elevados para el equipo que lo utiliza.
		\item Gasta batería en consecuencia.
	\end{enumerate}
\end{itemize}

\section{Automatización de pruebas}

\subsection{Selenium}

\begin{itemize}
	\item \textbf{Nombre de la herramienta}: Selenium
	\item \textbf{URL}: \url{https://www.seleniumhq.org/}
	\item \textbf{Coste}: Gratuito.
	\item \textbf{Características destacadas}:
	\begin{enumerate}
		\item Viene con una serie de herramientas ya integradas.
		\item Ofrece distintas soluciones para atender distintos requisitos.
		\item Uso extensible por medio de plugins.
	\end{enumerate}
	\item \textbf{Limitaciones identificadas}:
	\begin{enumerate}
		\item Solo se puede utilizar para probar aplicaciones de escritorio o otro tipo de software, solo web. 
		\item No hay soporte garantizado disponible para Selenium, hay que apoyarse en la comunidad.
		\item No es posible realizar pruebas en imágenes.
	\end{enumerate}
\end{itemize}

\subsection{Apache JMeter}

\begin{itemize}
	\item \textbf{Nombre de la herramienta}: Apache JMeter
	\item \textbf{URL}: \url{https://jmeter.apache.org/}
	\item \textbf{Coste}: Gratuito.
	\item \textbf{Características destacadas}:
	\begin{enumerate}
		\item Capacidad de carga y prueba de rendimiento de muchos tipos diferentes de aplicaciones/servidores/protocolos.
		\item Se pueden hacer usos de comandos nativos o scripts de shell.
		\item JMeter puede generar informes robustos y efectivos y puede ser visualizado con una gran cantidad de herramientas.
	\end{enumerate}
	\item \textbf{Limitaciones identificadas}:
	\begin{enumerate}
		\item JMeter aún no tiene complemento de autocorrelación.
		\item Todas las instancias de JMeter deben cerrarse mientras se ejecuta una prueba de carga.
		\item En ocasiones, si al copiar un elemento utilizamos versiones de JMeter distintas, se producirán errores.
	\end{enumerate}
\end{itemize}

\subsection{JUnit}

\begin{itemize}
	\item \textbf{Nombre de la herramienta}: JUnit
	\item \textbf{URL}: \url{https://junit.org/junit5/}
	\item \textbf{Coste}: Gratuito.
	\item \textbf{Características destacadas}:
	\begin{enumerate}
		\item Integración con Android Studio.
		\item Gran cantidad de documentación.
		\item Soportado en gran cantidad de IDEs.
	\end{enumerate}
	\item \textbf{Limitaciones identificadas}:
	\begin{enumerate}
		\item Exclusivo para Java.
		\item Exclusivo para test unitarios.
		\item Encontramos limitaciones en su funcionalidad.
	\end{enumerate}
\end{itemize}

\section{Integración continua}

\subsection{Travis IC}

\begin{itemize}
	\item \textbf{Nombre de la herramienta}: Travis IC
	\item \textbf{URL}: \url{https://travis-ci.org/}
	\item \textbf{Coste}: Gratuito para repositorios git públicos. Para empresas, tenemos precios desde 69\$ el más básico hasta 489\$ por mes.	
	\item \textbf{Características destacadas}:
	\begin{enumerate}
		\item Facilidad de integración con repositorios git.
		\item Fácil de utilizar.
		\item Se puede vincular a correo electrónico para ver los resultados.
	\end{enumerate}
	\item \textbf{Limitaciones identificadas}:
	\begin{enumerate}
		\item Precio elevado.
		\item La velocidad no es muy grande.
		\item Versión gratuita exclusiva a repositorios públicos.
	\end{enumerate}
\end{itemize}

\subsection{Bamboo}

\begin{itemize}
	\item \textbf{Nombre de la herramienta}: Bamboo
	\item \textbf{URL}: \url{https://es.atlassian.com/software/bamboo}
	\item \textbf{Coste}: Tiene prueba gratuita de 30 días con número de usuarios ilimitados. Después son 10\$ al año hasta 10 trabajos o 1100\$ con trabajos ilimitados.
	\item \textbf{Características destacadas}:
	\begin{enumerate}
		\item Disponible para cualquier lenguaje.
		\item Puede ejecutar múltiples construcciones en paralelo.
		\item Proporciona una API REST, esta proporciona información sobre el servidor, el estado actual de sus compilaciones, etc.
	\end{enumerate}
	\item \textbf{Limitaciones identificadas}:
	\begin{enumerate}
		\item Márgenes de precios demasiado extremos, no podemos ajustar precio a necesidades concretas.
		\item Si hacemos uso de Perforce, no se podrán ejecutar compilaciones en varios agentes remotos.
		\item Complejo de usar.
	\end{enumerate}
\end{itemize}

\subsection{Jira}

\begin{itemize}
	\item \textbf{Nombre de la herramienta}: Jira
	\item \textbf{URL}: \url{https://es.atlassian.com/software/jira}
	\item \textbf{Coste}: Prueba gratuita durante 10 días. Después son 10\$ al mes hasta 10 usuarios o 7\$ al mes por cada usuario cuando son entre 11 y 100. 
	\item \textbf{Características destacadas}:
	\begin{enumerate}
		\item La asignación es granular.
		\item Encontramos dos niveles de issues (tareas): issues y sub-issues.
		\item Posee una gran cantidad de herramientas en torno a él.
	\end{enumerate}
	\item \textbf{Limitaciones identificadas}:
	\begin{enumerate}
		\item Difícil de instalar.
		\item Difícil de usar.
		\item Excesivas características.
	\end{enumerate}
\end{itemize}

\section{Seguimiento de errores/defectos}	 

\subsection{Sentry}

\begin{itemize}
	\item \textbf{Nombre de la herramienta}: Sentry
	\item \textbf{URL}: \url{https://sentry.io/welcome/}
	\item \textbf{Coste}: Gratuito.
	\item \textbf{Características destacadas}:
	\begin{enumerate}
		\item Impacto de despliegues en tiempo real.
		\item Nos proporciona ayuda en ciertos momentos.
		\item Fácil integración externa.
	\end{enumerate}
	\item \textbf{Limitaciones identificadas}:
	\begin{enumerate}
		\item Documentación incompleta.
		\item Problemas de seguridad.
		\item Requiere una dependencia binario para poder operar.
	\end{enumerate}
\end{itemize}

\subsection{GitLab}

\begin{itemize}
	\item \textbf{Nombre de la herramienta}: GitLab
	\item \textbf{URL}: \url{https://about.gitlab.com/}
	\item \textbf{Coste}: Gratuito.
	\item \textbf{Características destacadas}:
	\begin{enumerate}
		\item Buena curva de aprendizaje.
		\item Amplia comunidad.
		\item Robusto y flexible.
	\end{enumerate}
	\item \textbf{Limitaciones identificadas}:
	\begin{enumerate}
		\item Almacenamiento del repositorio limitado a 10GB.
		\item Interfaz relativamente lenta.
		\item Frecuentes problemas técnicos con los repositorios.
	\end{enumerate}
\end{itemize}

\subsection{Mantis Bug Tracker}

\begin{itemize}
	\item \textbf{Nombre de la herramienta}: Mantis Bug Tracker.
	\item \textbf{URL}: \url{https://www.mantisbt.org/}
	\item \textbf{Coste}: Gratuito.
	\item \textbf{Características destacadas}:
	\begin{enumerate}
		\item Se permite configurar la transición de estados.
		\item Se puede especificar un número indeterminado de estados.
		\item Permite introducir diferentes perfiles.
	\end{enumerate}
	\item \textbf{Limitaciones identificadas}:
	\begin{enumerate}
		\item Falta de documentación.
		\item Interfaz anticuada.
		\item Menos funcionalidad que otras herramientas.
	\end{enumerate}
\end{itemize}


\section{Comparativa de herramientas}

En esta sección se va a realizar una comparativa de herramientas, las cuales serán valoradas del 0 al 1, de tal forma que al multiplicar por los pesos siempre salgan valores entre 0 y 1. Destacar un criterio que será común en todas las tablas comparativas; el precio. Inicialmente, el proyecto se buscó que fuese económico y, por tanto, este criterio es especialmente importante.

\subsection{Control de versiones}

\begin{table}[H]
	\begin{center}
		\begin{tabular}{|c||c|c|c|c|}
			\hline
			Criterio & Peso & Git & Bazaar & Mercurial \\
			\hline \hline
			Precio & 0.4 & 1 & 1 & 1 \\ \hline
			Robustez & 0.3 & 0.6 & 0.45 & 0.6 \\ \hline
			Estabilidad & 0.3 & 0.7 & 0.5 & 0.7 \\ \hline
			\textbf{Total} & \textbf{1} & \textbf{0.79} & \textbf{0.685} & \textbf{0.79} \\ \hline
		\end{tabular}
		\caption{Tabla comparativa de herramientas de control de versiones.}
		\label{tabla:tabla1}
	\end{center}
\end{table}

\subsection{Compilación}

\begin{table}[H]
	\begin{center}
		\begin{tabular}{|c||c|c|c|c|}
			\hline
			Criterio & Peso & Eclipse & NetBeans & Android Studio \\
			\hline \hline
			Precio & 0.4 & 1 & 0.7 & 1 \\ \hline
			Rapidez & 0.3 & 0.25 & 0.25 & 0.8 \\ \hline
			Uso de recursos & 0.3 & 0.6 & 0.5 & 0.2 \\ \hline
			\textbf{Total} & \textbf{1} & \textbf{0.655} & \textbf{0.505} & \textbf{0.7} \\ \hline
		\end{tabular}
		\caption{Tabla comparativa de herramientas de compilación.}
		\label{tabla:tabla2}
	\end{center}
\end{table}

\subsection{Automatización de pruebas}

\begin{table}[H]
	\begin{center}
		\begin{tabular}{|c||c|c|c|c|}
			\hline
			Criterio & Peso & Selenium & Apache JMeter & JUnit \\
			\hline \hline
			Precio & 0.4 & 1 & 1 & 1 \\ \hline
			Integración & 0.35 & 0.4 & 0.7 & 0.8 \\ \hline
			Velocidad & 0.25 & 0.6 & 0.3 & 0.7 \\ \hline
			\textbf{Total} & \textbf{1} & \textbf{0.69} & \textbf{0.72} & \textbf{0.855} \\ \hline
		\end{tabular}
		\caption{Tabla comparativa de herramientas de automatización de pruebas.}
		\label{tabla:tabla3}
	\end{center}
\end{table}


\subsection{Integración continua}

\begin{table}[H]
	\begin{center}
		\begin{tabular}{|c||c|c|c|c|}
			\hline
			Criterio & Peso & Travis IC & Bamboo & Jira \\
			\hline \hline
			Precio & 0.4 & 0.5 & 0.6 & 0.5 \\ \hline
			Velocidad & 0.5 & 0.3 & 0.75 & 0.45 \\ \hline
			Usabilidad & 0.1 & 0.9 & 0.2 & 0.1 \\ \hline
			\textbf{Total} & \textbf{1} & \textbf{0.44} & \textbf{0.635} & \textbf{0.435} \\ \hline
		\end{tabular}
		\caption{Tabla comparativa de herramientas de integración continua.}
		\label{tabla:tabla4}
	\end{center}
\end{table}


\subsection{Seguimiento de errores/defectos}

\begin{table}[H]
	\begin{center}
		\begin{tabular}{|c||c|c|c|c|}
			\hline
			Criterio & Peso & Sentry & GitLab & Mantis Bug Tracker \\
			\hline \hline
			Precio & 0.4 & 1 & 1 & 1 \\ \hline
			Comunidad y soporte & 0.2 & 0.85 & 0.8 & 0.4 \\ \hline
			Funcionalidad & 0.4 & 0.15 & 0.5 & 0.3 \\ \hline
			\textbf{Total} & \textbf{1} & \textbf{0.63} & \textbf{0.76} & \textbf{0.6} \\ \hline
		\end{tabular}
		\caption{Tabla comparativa de herramientas de seguimiento de errores/defectos.}
		\label{tabla:tabla5}
	\end{center}
\end{table}




%\newpage
%\bibliographystyle{plain}
%\bibliography{biblio}

\end{document}       
%---------------------------------------------------
