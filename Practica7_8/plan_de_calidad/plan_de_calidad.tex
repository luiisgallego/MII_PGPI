%%%%%%%%%%%%%%%%%%%%%%%%%%%%%%%%%%%%%%%%%
% Short Sectioned Assignment LaTeX Template Version 1.0 (5/5/12)
% This template has been downloaded from: http://www.LaTeXTemplates.com
% Original author:  Frits Wenneker (http://www.howtotex.com)
% License: CC BY-NC-SA 3.0 (http://creativecommons.org/licenses/by-nc-sa/3.0/)
%%%%%%%%%%%%%%%%%%%%%%%%%%%%%%%%%%%%%%%%%

%----------------------------------------------------------------------------------------
%	PACKAGES AND OTHER DOCUMENT CONFIGURATIONS
%----------------------------------------------------------------------------------------

\documentclass[paper=a4, fontsize=11pt]{scrartcl} % A4 paper and 11pt font size

% ---- Entrada y salida de texto -----

\usepackage[T1]{fontenc} % Use 8-bit encoding that has 256 glyphs
\usepackage[utf8]{inputenc}
%\usepackage{fourier} % Use the Adobe Utopia font for the document - comment this line to return to the LaTeX default

% ---- Idioma --------

\usepackage[spanish, es-tabla]{babel} % Selecciona el español para palabras introducidas automáticamente, p.ej. "septiembre" en la fecha y especifica que se use la palabra Tabla en vez de Cuadro

% ---- Otros paquetes ----

\usepackage{url} % ,href} %para incluir URLs e hipervínculos dentro del texto (aunque hay que instalar href)
\usepackage{amsmath,amsfonts,amsthm} % Math packages
%\usepackage{graphics,graphicx, floatrow} %para incluir imágenes y notas en las imágenes
\usepackage{graphics,graphicx, float} %para incluir imágenes y colocarlas

% Para hacer tablas comlejas
%\usepackage{multirow}
%\usepackage{threeparttable}

%\usepackage{sectsty} % Allows customizing section commands
%\allsectionsfont{\centering \normalfont\scshape} % Make all sections centered, the default font and small caps

\usepackage{fancyhdr} % Custom headers and footers
\pagestyle{fancyplain} % Makes all pages in the document conform to the custom headers and footers
\fancyhead{} % No page header - if you want one, create it in the same way as the footers below
\fancyfoot[L]{} % Empty left footer
\fancyfoot[C]{} % Empty center footer
\fancyfoot[R]{\thepage} % Page numbering for right footer
\renewcommand{\headrulewidth}{0pt} % Remove header underlines
\renewcommand{\footrulewidth}{0pt} % Remove footer underlines
\setlength{\headheight}{13.6pt} % Customize the height of the header

\numberwithin{equation}{section} % Number equations within sections (i.e. 1.1, 1.2, 2.1, 2.2 instead of 1, 2, 3, 4)
\numberwithin{figure}{section} % Number figures within sections (i.e. 1.1, 1.2, 2.1, 2.2 instead of 1, 2, 3, 4)
\numberwithin{table}{section} % Number tables within sections (i.e. 1.1, 1.2, 2.1, 2.2 instead of 1, 2, 3, 4)

\setlength\parindent{0pt} % Removes all indentation from paragraphs - comment this line for an assignment with lots of text

\newcommand{\horrule}[1]{\rule{\linewidth}{#1}} % Create horizontal rule command with 1 argument of height

%	TÍTULO Y DATOS DEL ALUMNO
%----------------------------------------------------------------------------------------

\title{	
	\normalfont \normalsize 
	\textsc{\textbf{Planificación y Gestión de Proyectos Informáticos (2018-2019)} \\ Máster Profesional de Ingeniería Informática \\ Universidad de Granada} \\ [25pt] % Your university, school and/or department name(s)
	\horrule{0.5pt} \\[0.4cm] % Thin top horizontal rule
	\huge Plan de Calidad \\ % The assignment title
	\horrule{2pt} \\[0.5cm] % Thick bottom horizontal rule
}

\author{Alejandro Campoy Nieves \\ Luis Gallego Quero} % Nombre y apellidos y correo
\date{\normalsize\today} % Incluye la fecha actual

\usepackage[spanish, es-tabla]{babel}
\usepackage{hyperref} % Para añadir los hiperenlaces.
\hypersetup{
	colorlinks=true,
	linkcolor=blue,
	filecolor=magenta,      
	urlcolor=blue,
}
\usepackage{graphicx}
\usepackage{amssymb, amsmath, amsbsy}
\usepackage{mathptmx}	
\usepackage{float}
\usepackage{booktabs}					%paquete para realización de tablas profesionales
\usepackage{eurosym}

\usepackage[table]{xcolor}
\usepackage{color}
\usepackage{colortbl}
\usepackage{multicol}
\usepackage{multirow}
\usepackage{booktabs}
\usepackage{tabularx}
%---- Paquete para aliniación de texto verticalmente dentro de tablas
\usepackage{array}
%---- Paquetes para los pies de las tablas
\usepackage{caption}
\usepackage{subcaption}


%----------------------------------------------------------------------------------------
% DOCUMENTO
%----------------------------------------------------------------------------------------

\begin{document}
	\maketitle % Muestra el Título
	
	\newpage %inserta un salto de página
	
	\tableofcontents % para generar el índice de contenidos
	
	%\listoffigures
	
	%\listoftables	
	
	\newpage	
 
\section{Introducción}

En el presente documentos vamos a mostrar las distintas peticiones de cambio que han surgido durante el desarrollo, siendo estas las indicadas en el pdf de la práctica y proporcionaremos un plan de acción destinado a cada petición.

\section{Peticiones de cambio}

\begin{table}[H]
	\begin{center}
		\begin{tabular}{|c|c|} \toprule
			Tipo & Petición de cambio \\ \midrule
			Recursos humanos & Se necesita contratar durante dos meses a un programador extra.    \\
			& No afecta al tiempo de entrega. \\ 
			Recursos materiales & Se necesita comprar un servidor muy potente para  \\
			& dar soporte a la administración de usuarios o procesamiento de datos. \\
			Tiempo: retraso & Las actividades relacionadas con la implementación  \\
			& de la aplicación móvil durarán dos semanas más de lo  \\
			& planificado inicialmente. \\
			Costes: aumento &  El sueldo mensual de los trabajadores se incrementa en un 2\%. \\
			Requisitos &  La aplicación web y móvil deben tener los colores \\
			& y logos del cliente, lo exige la normativa. \\
			Requisitos & Hay un nuevo requisito funcional por parte del cliente: registro  \\
			& de qué elementos del museo visitan para su uso en sondeos. \\
			Diseño, Metodología & Se va a seguir una metodología de Ingeniería Web   \\
			& para el diseñol de la aplicación web del administrador. \\
			Alcance, Metodología & Debemos supervisar en cada momento \\
			& si los objetivos del proyecto se están cumpliendo o no. \\
			& Estaba previsto hacerlo solo al final del proyecto. \\
			Pruebas o incidencias, Metodología & Se va a crear un sub-equipo específico  \\ 
			& para gestionar las incidencias de las aplicaciones \\
			& creadas e instaladas por la empresa de desarrollo. \\ \bottomrule
		\end{tabular}
	\end{center}
	\caption{Diferentes peticiones de cambio}
	\label{peticionesCambio}
\end{table}

\section{Plan de acción}

\subsection{Plan de acción 1}

\begin{itemize}
	\item \textbf{Qué se hará: } Se ampliará la plantilla con alguien que tenga un nivel alto de conocimientos sobre las tecnologías usadas en el proyecto. 
	\item \textbf{Por qué: } Para no dedicar una cantidad de tiempo grande en la adaptación del nuevo programador al proyecto.
	\item \textbf{Quién: } El encargado de recursos humanos.
	\item \textbf{Cuándo: } Lo antes posible.
	\item \textbf{Consecuencias: } Como incluimos un nuevo sueldo a la plantilla, los costes aumentan.
\end{itemize}

\subsection{Plan de acción 2}

\begin{itemize}
	\item \textbf{Qué se hará: } Buscar las mejores ofertas en este recurso. Principalmente nos centraremos en la compra de un nuevo servidor.
	\item \textbf{Por qué: } Principalmente porque vamos a procesar gran cantidad de datos , por lo que a la larga será más económico comprar un servidor, que subcontratarlo en función de las horas que lo usemos.
	\item \textbf{Quién: } El más especializado en este ámbito en el equipo.
	\item \textbf{Cuándo: } En el momento que tengamos que realizar ese procesado de datos.
	\item \textbf{Consecuencias: } Los costes aumentan de nuevo, además hay que tener en cuenta el coste de mantenimiento e implantación.
\end{itemize}

\subsection{Plan de acción 3}

\begin{itemize}
	\item \textbf{Qué se hará: }
	\item \textbf{Por qué: }
	\item \textbf{Quién: }
	\item \textbf{Cuándo: }
	\item \textbf{Consecuencias: }
\end{itemize}

\subsection{Plan de acción 4}

\begin{itemize}
	\item \textbf{Qué se hará: }
	\item \textbf{Por qué: }
	\item \textbf{Quién: }
	\item \textbf{Cuándo: }
	\item \textbf{Consecuencias: }
\end{itemize}

\subsection{Plan de acción 5}

\begin{itemize}
	\item \textbf{Qué se hará: }
	\item \textbf{Por qué: }
	\item \textbf{Quién: }
	\item \textbf{Cuándo: }
	\item \textbf{Consecuencias: }
\end{itemize}

\subsection{Plan de acción 6}

\begin{itemize}
	\item \textbf{Qué se hará: }
	\item \textbf{Por qué: }
	\item \textbf{Quién: }
	\item \textbf{Cuándo: }
	\item \textbf{Consecuencias: }
\end{itemize}

\subsection{Plan de acción 7}

\begin{itemize}
	\item \textbf{Qué se hará: }
	\item \textbf{Por qué: }
	\item \textbf{Quién: }
	\item \textbf{Cuándo: }
	\item \textbf{Consecuencias: }
\end{itemize}

\subsection{Plan de acción 8}

\begin{itemize}
	\item \textbf{Qué se hará: }
	\item \textbf{Por qué: }
	\item \textbf{Quién: }
	\item \textbf{Cuándo: }
	\item \textbf{Consecuencias: }
\end{itemize}

\subsection{Plan de acción 8}

\begin{itemize}
	\item \textbf{Qué se hará: }
	\item \textbf{Por qué: }
	\item \textbf{Quién: }
	\item \textbf{Cuándo: }
	\item \textbf{Consecuencias: }
\end{itemize}



%\newpage
%\bibliographystyle{plain}
%\bibliography{biblio}

\end{document}       
%---------------------------------------------------
