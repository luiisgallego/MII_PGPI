%%%%%%%%%%%%%%%%%%%%%%%%%%%%%%%%%%%%%%%%%
% Short Sectioned Assignment LaTeX Template Version 1.0 (5/5/12)
% This template has been downloaded from: http://www.LaTeXTemplates.com
% Original author:  Frits Wenneker (http://www.howtotex.com)
% License: CC BY-NC-SA 3.0 (http://creativecommons.org/licenses/by-nc-sa/3.0/)
%%%%%%%%%%%%%%%%%%%%%%%%%%%%%%%%%%%%%%%%%

%----------------------------------------------------------------------------------------
%	PACKAGES AND OTHER DOCUMENT CONFIGURATIONS
%----------------------------------------------------------------------------------------

\documentclass[paper=a4, fontsize=11pt]{scrartcl} % A4 paper and 11pt font size

% ---- Entrada y salida de texto -----

\usepackage[T1]{fontenc} % Use 8-bit encoding that has 256 glyphs
\usepackage[utf8]{inputenc}
%\usepackage{fourier} % Use the Adobe Utopia font for the document - comment this line to return to the LaTeX default

% ---- Idioma --------

\usepackage[spanish, es-tabla]{babel} % Selecciona el español para palabras introducidas automáticamente, p.ej. "septiembre" en la fecha y especifica que se use la palabra Tabla en vez de Cuadro

% ---- Otros paquetes ----

\usepackage{url} % ,href} %para incluir URLs e hipervínculos dentro del texto (aunque hay que instalar href)
\usepackage{amsmath,amsfonts,amsthm} % Math packages
%\usepackage{graphics,graphicx, floatrow} %para incluir imágenes y notas en las imágenes
\usepackage{graphics,graphicx, float} %para incluir imágenes y colocarlas

% Para hacer tablas comlejas
%\usepackage{multirow}
%\usepackage{threeparttable}

%\usepackage{sectsty} % Allows customizing section commands
%\allsectionsfont{\centering \normalfont\scshape} % Make all sections centered, the default font and small caps

\usepackage{fancyhdr} % Custom headers and footers
\pagestyle{fancyplain} % Makes all pages in the document conform to the custom headers and footers
\fancyhead{} % No page header - if you want one, create it in the same way as the footers below
\fancyfoot[L]{} % Empty left footer
\fancyfoot[C]{} % Empty center footer
\fancyfoot[R]{\thepage} % Page numbering for right footer
\renewcommand{\headrulewidth}{0pt} % Remove header underlines
\renewcommand{\footrulewidth}{0pt} % Remove footer underlines
\setlength{\headheight}{13.6pt} % Customize the height of the header

\numberwithin{equation}{section} % Number equations within sections (i.e. 1.1, 1.2, 2.1, 2.2 instead of 1, 2, 3, 4)
\numberwithin{figure}{section} % Number figures within sections (i.e. 1.1, 1.2, 2.1, 2.2 instead of 1, 2, 3, 4)
\numberwithin{table}{section} % Number tables within sections (i.e. 1.1, 1.2, 2.1, 2.2 instead of 1, 2, 3, 4)

\setlength\parindent{0pt} % Removes all indentation from paragraphs - comment this line for an assignment with lots of text

\newcommand{\horrule}[1]{\rule{\linewidth}{#1}} % Create horizontal rule command with 1 argument of height


%----------------------------------------------------------------------------------------
%	TÍTULO Y DATOS DEL ALUMNO
%----------------------------------------------------------------------------------------

\title{	
	\normalfont \normalsize 
	\textsc{\textbf{Planificación y Gestión de Proyectos Informáticos (2018-2019)} \\ Máster Profesional de Ingeniería Informática \\ Universidad de Granada} \\ [25pt] % Your university, school and/or department name(s)
	\horrule{0.5pt} \\[0.4cm] % Thin top horizontal rule
	\huge Manual de Coordinación \\ % The assignment title
	\horrule{2pt} \\[0.5cm] % Thick bottom horizontal rule
}

\author{Alejandro Campoy Nieves \\ Luis Gallego Quero} % Nombre y apellidos y correo
\date{\normalsize\today} % Incluye la fecha actual
\usepackage{graphicx}
\usepackage{hyperref} % Para añadir los hiperenlaces.


%----------------------------------------------------------------------------------------
% DOCUMENTO
%----------------------------------------------------------------------------------------

\begin{document}
\maketitle % Muestra el Título

\newpage %inserta un salto de página

\tableofcontents % para generar el índice de contenidos

%\listoffigures

%\listoftables

\newpage		
 
\section{Ciclo de vida}

El ciclo de vida que vamos a utilizar es iterativo. Se trata de ir obteniendo parte del producto por pequeños bloques a los que se les denominan ciclos de desarrollo o iteraciones. La razón de esto es ir obteniendo desde el primer momento un proyecto funcional. Aunque en las primeras etapas el programa no cumple las expectativas esperadas y no sea fiable dando resultados de calidad dudosa, tenemos la posibilidad de obtener una aplicación útil en la que podemos ir viendo los resultados iniciales, valorarlos y pensar o analizar la estrategia para la siguiente iteración.

\section{Metodología de desarrollo}

Ligado al punto anterior, utilizaremos una metodología ágil basada en SCRUM. Cuyo objetivo sea gestionar y planificar el proyecto con posibilidades de cambio a última hora. 

\section{Recursos software desarrollo}

Desarrollaremos este proyecto a través de Spyder, un IDE para Python que nos ayudará al desarollo de este sistema. Incluyendo las librerías y funcionalidades disponibles que nos permitan tener una base en la que montar este proyecto y que nos aporte una mejora en la calidad y facilidad de trabajo. \\

Para la realización de los distintos diagramas utilizaremos las herramientas que nos proporciona Draw.io, Microsoft Visio y github como gestor de documentación y controlador de versiones.

\section{Organización del equipo de trabajo}

La organización que hemos consensuado es de tipo dinámico. Esto quiere decir, que los integrantes de este proyecto desarrollan roles temporales durante un intervalo variable. En resumen, una persona se dedica a supervisar el proyecto principalmente, mientras que la otra lo desarrolla. Estos papeles se intercambian con el objetivo de no saturar las capacidades de trabajo de cada uno.

\section{Herramientas para comunicaciones en el equipo de trabajo}

\begin{itemize}
	\item Github como controlador de versiones.
	\item Trello para el desarrollo de SCRUM.
	\item Whatsapp para la comunicación entre los integrantes del proyecto.
	\item Hangouts para posibles videoconferencias.
\end{itemize}

\section{Relaciones con el cliente}

\begin{enumerate}
	\item Entrevista: Primer encuentro con el cliente destinado en esencia al análisis de requisitos del proyecto que abordamos. 
	\item Reuniones: Siendo la base de SCRUM realizar reuniones periódicas con el cliente, de este modo, podemos presentarle los diferentes avances y, en ese momento, se puede perfilarse distintos aspectos del proyecto con el objetivo de ajustarse a las necesidades del cliente en cuestión.
\end{enumerate}

\section{Estándares de documentación}

\begin{itemize}
	\item Como compositor de texto, utilizaremos la herramienta llamada \LaTeX.
	\item De portada utilizaremos siempre la misma plantilla para darle homogeneidad al proyecto.
	\item Se enumeraran las páginas de forma automática sin contar la portada, de tal forma que se desarrolla un índice acorde a cada parte de la documentación.
	\item Aunque el texto se escriba en formato \LaTeX, la documentación final del proyecto será presentada en el formato PDF.
\end{itemize}

\section{estándares de código}

\begin{itemize}
	\item Como vamos a realizar el proyecto en Python, utilizaremos los convenios de presentación típicos de este lenguaje (en las variables, primera letra minuscula y resto de palabras primera letra en mayúscula, por ejemplo).
	\item Los comentarios serán especificados en castellano siguiendo el formato de comentarios de Python para los distinto módulos.
\end{itemize}


\section{Control de versiones}

Tanto para los documentos como el código generado durante el desarrollo de este proyecto, serán almacenados, controlados y dirigidos a través de Github.

\section{Gestión de calidad}

Los test para realizar las correspondientes pruebas de desarrollo serán elaboradas a través de la herramienta de Python llamada Unittest.\\

También se utilizará como punto de apoyo, las verificaciones y visto bueno por parte del cliente, que será considerado como un punto más del proceso de prueba y testeo.

%\bibliographystyle{plain}
%\bibliography{biblio}

\end{document}       
%---------------------------------------------------
